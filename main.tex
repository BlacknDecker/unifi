\documentclass[a4paper, 12pt]{article}

\usepackage{hyperref}
\usepackage{xcolor}
\usepackage{graphicx}
\usepackage[italian]{babel}
\usepackage[T1]{fontenc}
\usepackage{url}
\usepackage{import}
\usepackage{multirow}
\usepackage{color}
\usepackage{fancyhdr}
\usepackage{amssymb}
\usepackage{tabu}
\usepackage{mathtools}
\usepackage[margin=2.5cm]{geometry}
\usepackage{listings}
\usepackage{titling}
\usepackage[utf8]{inputenc}
\usepackage[numbered,framed]{matlab-prettifier}

\let\ph\mlplaceholder% shorter macro
\lstMakeShortInline"

\newcommand{\bnb}{\begin{nobreak}}
\newcommand{\enb}{\end{nobreak}}

\lstset{
  style              = Matlab-editor,
  basicstyle         = \mlttfamily,
  escapechar         = ",
  mlshowsectionrules = true,
}

\pretitle{%
  \begin{center}
  \huge
  \includegraphics[width=\textwidth/4]{logo_unifi.png}\\[\bigskipamount]
}
\posttitle{\end{center}}
\date{}

\begin{document}

\title{
	\vspace{3cm}
	\textbf{Calcolo Numerico}
	\\
	a.a. 2017/2018
	\vspace{1.5cm}
}

\author{
	Simone \textbf{Cipriani}, \texttt{5951907} --- \href{mailto:sim.cipr@gmail.com}{\textit{sim.cipr@gmail.com}}
	\vspace{1cm}
	\\
	Repository git del progetto hostata su GitHub:
	\\
	\url{https://github.com/gesucca/cn-class-project.git}
}

\maketitle
\newpage
\tableofcontents


\newpage
\section{\textbf{Capitolo 1}}
\subsection{\textbf{Esercizio 1.1}}
\vspace{0.5cm}
Dato \(\overline{x} = 2.7182\) e dato \(x = e = 2.7182818284\ldots \)

\begin{itemize}

\item errore assoluto: \(\Delta x = \overline{x} - x = 2.7182 - e \approx - 818 \times 10^{-7}\)
\item errore relativo: \(\epsilon_x = \frac{\Delta x}{x} = \frac{- 818 \times 10^{-7}}{e} = - 3 \times 10^{-5}\)

\end{itemize}

\noindent Si verfica adesso che:

\begin{itemize}
\item \(\overline{x} = 2.7182\) ha 4 cifre decimali
\item \(- \log_{10}|\epsilon_x| \approx 4.5 \)
\end{itemize}



\newpage
\section{\textbf{Capitolo 2}}
\vspace{0.8cm}
\subsection{\textbf{Esercizio 1}}
\vspace{0.5cm}
\begin{center}
\footnotesize\noindent\fbox{
	\parbox{\textwidth}{
	Determinare analiticamente gli zeri del polinomio
	\[
	P(x) = x^3 - 4x^2 + 5x - 2
	\]
	e la loro molteplicit\'a. Dire perch\'e il metodo i bisezione \'e utilizzabile per approssimarne uno a partire dall'intervallo di confidenza \([a,b] = [0,3]\). A quale zero di \(P\) potr\'a tendere la successione generata dal metodo di bisezione a partire da tale intervallo? Costruire una tabella i cui si riportano il numero di iterazioni e di valutazioni di \(P\) richieste per valori decrescenti della tolleranza \(tolx\).
	}
}\end{center}

\noindent Dato \(P(x) = x^3 - 4x^2 + 5x - 2\), si determinano anzitutto le sue radici analiticamente:

\[
P(x) = x^3 - 4x^2 + 5x - 2\ = (x-2)(x-1)^2
\]

\noindent Gli zeri di \(P(x)\) sono quindi \(x_1=1\) con molteplicit\'a \(2\) e \(x_2=2\) con molteplicit\'a \(1\).\\

\noindent Il metodo di bisezione \'e utilizzabile in quanto le ipotesi necessarie per il suo impiego sono soddisfatte:

\begin{itemize}

\item \(f(x)\) associata a \(P(x)\) \'e definita e continua nell'intervallo \([0,3]\)
\item \(f(0)f(3) = (-2)(4) = -8 < 0\)

\end{itemize}

\noindent Per conoscere a quale zero di \(P(x)\) tender\'a la successione generata dal metodo di bisezione, basta eseguire manualmente la prima iterazione:

\begin{itemize}
\item \(f(\frac{3-0}{2}) = -\frac{1}{8}\): la soluzione del problema non \'e \(\frac{3}{2}\)
\item \(f(0)f(\frac{3}{2}) = \frac{1}{4}\)
\item \(f(\frac{3}{2})f(3) = -\frac{1}{2} \rightarrow \) si prosegue ad iterare sull'intervallo \((\frac{3}{2}, 3)\)
\end{itemize}

\noindent Dato che si prosegue la bisezione nell'intervallo \([\frac{3}{2}, 3]\), si tender\'a allo zero in \(x_2=2\).

\noindent Si applica adesso il metodo di bisezione con il seguente nella sua interezza e si mostra il numero di iterazioni necessarie al variare della soglia di tolleranza \(tolx\).\\

\begin{tabular}{l*{6}{c}}
 \(tolx\) & appross. \(x_2\) & iterazioni \\
\hline
 1/10 & 1.8750 & 3\\
 1/20 & 1.9688 & 5\\
 1/50 & 2.0156 & 6\\
 1/100 & 1.9922 & 7\\
 1/500 & 1.9980 & 9\\
 \(1 \times 10^{-3}\) & 2.0010 & 10\\
 \(2 \times 10^{-3}\) & 1.9995 & 11\\
 \(5 \times 10^{-3}\) & 1.9999 & 13\\
 \(1 \times 10^{-4}\) & 2.0001 & 14\\
 \(1 \times 10^{-5}\) & 2.0000 & 17\\
 \(1 \times 10^{-6}\) & 2.0000 & 20\\
\end{tabular} \\

\noindent La precedente tabella \'e stata riempita in riferimento alla seguente implementazione del metodi di bisezione, descritta in codice Matlab.\\

\lstinputlisting[language=Matlab]{cap2/es1.m}

\vspace{1cm}
\subsection{\textbf{Esercizio 2}}
\vspace{0.5cm}
Si faccia ancora riferimento ad \(f(x)\) relativa a \(P(x)=x^3 - 4x^2 + 5x - 2\) con radici in \(x_1=1\) e \(x_2=2\) e la sua derivata prima \(f'(x) = 3x^2 - 8x + 5\). Si estende di seguito la tabella dell'esercizio precedente, aggiungendo i risultati ottenuti lanciando delle implementazioni dei metodi di Newton, delle corde e delle secanti a partire dal punto \(x_0 = 3\).

\begin{tabular}{l*{12}{c}}
 & \vline& \textbf{Bisez.} & & \vline& \textbf{Newton} & & \vline& \textbf{Corde} & & \vline& \textbf{Secanti} \\
 \(tolx\) & \vline& appr. \(x_2\) & it. & \vline& appr. \(x_2\) & it.& \vline& appr. \(x_2\) & it.& \vline& appr. \(x_2\) & it.\\
\hline
 1/10 & \vline& 1.8750 & 3 & \vline& 2.0043 & 3 & \vline& 2.3594 & 1 & \vline& 2.0502 & 3\\
 1/20 & \vline& 1.9688 & 5 & \vline& '' & ''& \vline& 2.2201 & 3 & \vline& 2.0108 & 4\\
 1/50 & \vline& 2.0156 & 6 & \vline& 2.0000 & 4 & \vline& 2.1236 & 6 & \vline& 2.0010 & 5\\
 1/100 & \vline& 1.9922 & 7 & \vline& '' &'' & \vline& 2.0643 & 10 & \vline& '' & ''\\
 1/500 & \vline& 1.9980 & 9 & \vline& '' & 5 & \vline& 2.0152 & 20 & \vline& 2.0000 & 6\\
 \(1 \times 10^{-3}\) & \vline& 2.0010 & 10 & \vline& '' & '' & \vline& 2.0077 & 25 & \vline& '' & ''\\
 \(2 \times 10^{-3}\) & \vline& 1.9995 & 11 & \vline& '' & '' & \vline& 2.0039 & 30 & \vline& '' & 7\\
 \(5 \times 10^{-3}\) & \vline& 1.9999 & 13 & \vline& '' & '' & \vline& 2.0015 & 37 & \vline& '' & ''\\
 \(1 \times 10^{-4}\) & \vline& 2.0001 & 14 & \vline& '' & '' & \vline& 2.0008 & 42 & \vline& '' & ''\\
 \(1 \times 10^{-5}\) & \vline& 2.0000 & 17 & \vline& '' & 6  & \vline& 2.0001 & 60 & \vline& '' & 8\\
 \(1 \times 10^{-6}\) & \vline& 2.0000 & 20 & \vline& '' & 6  & \vline& 2.0000 & 77 & \vline& '' & 8\\
\end{tabular} \\

\noindent QUALCHE COMMENTO SULLA CONVERGENZA\\

\noindent Nel merito della domanda finale posta nel testo dell'esercizio, non \'e possibile utilizzare come punto di innesco \(x_0=\frac{5}{3}\) in quanto \(\frac{5}{3}\) \'e uno zero della derivata prima, la cui valutazione in \(x_0\) sta al denominatore nel primo passo di tutti i metodi impiegati.\\

\noindent Di seguito le implementazioni Matlab dei metodi utilizzati.

\lstinputlisting[language=Matlab]{cap2/es2_functions.m}

\vspace{1cm}
\subsection{\textbf{Esercizio 3}}
\vspace{0.5cm}
\begin{center}
\footnotesize\noindent\fbox{
	\parbox{\textwidth}{
Costruire una seconda tabella analoga alla precedente relativa ai metodi di Newton, di Newton modificato e di accelerazione di Aitken applicati alla funzione polinomiale \(P\) a partire dal punto di innesco \(x_0=0\). Commentare i risultati riportati in tabela.
	}
}\end{center}

Si faccia ancora una volta riferimento ad \(f(x)\) relativa a \(P(x)=x^3 - 4x^2 + 5x - 2\) con radici in \(x_1=1\) e \(x_2=2\). Si presenta di seguito una tabella analoga, alla precedente, riferita ai metodi di Newton, di Newton modificato e di Aitken, lanciati con punto di innesco \(x_0 = 0\).\\

\begin{tabular}{l*{12}{c}}
 & \vline& \textbf{Newton} & & \vline& \textbf{N. Mod.} & & \vline& \textbf{Aitken} \\
 \(tolx\) & \vline& appr. \(x_1\) & it. & \vline& appr. \(x_1\) & it.& \vline& appr. \(x_1\) & it.\\
\hline
 1/10 & \vline& 0.8960 & 3						& \vline& 0.9961 & 2 & \vline& 1.0006 & 2 \\
 1/20 & \vline& 0.9457 & 4						& \vline& 0.1000 & 3 & \vline& ''     & ''\\
 1/50 & \vline& 0.9859 & 6						& \vline& ''     & 3 & \vline& 1.0000 & 3 \\
 1/100 & \vline& 0.9929& 7						& \vline& ''     & 3 & \vline& ''     & ''\\
 1/500 & \vline& 0.9982& 9						& \vline& ''     & 4 & \vline& ''     & ''\\
 \(1 \times 10^{-3}\) & \vline& 0.9991 & 10		& \vline& ''     & 4 & \vline& ''     & ''\\
 \(2 \times 10^{-3}\) & \vline& 0.9996 & 11		& \vline& ''     & 4 & \vline& ''     & 4\\
 \(5 \times 10^{-3}\) & \vline& 0.9999 & 13		& \vline& ''     & 4 & \vline& ''     & ''\\
 \(1 \times 10^{-4}\) & \vline& '' & 14			& \vline& ''     & 4 & \vline& ''     & ''\\
 \(1 \times 10^{-5}\) & \vline& 1.000 & 17		& \vline& ''     & 4 & \vline& ''     & ''\\
 \(1 \times 10^{-6}\) & \vline& '' & 20 		& \vline& ''     & 5 & \vline& ''     & ''\\
\end{tabular} \\

\noindent Come era prevedibile, il metodo di Newton standard converge linearmente invece che quadraticamente verso la soluzione a causa della molteplicit\'a della radice verso la quale si converge; il metodo di Newton modificato ha invece ripristinato la convergenza quadratica. DUE PAROLE SU AITKIN

\noindent Di seguito le implementazioni Matlab dei metodi numerici utilizzati. Per il metodo di Newton \'e stato sfruttata l'implementazione usata nell'esercizio precedente.

\lstinputlisting[language=Matlab]{cap2/es3_functions.m}

\vspace{1cm}
\subsection{\textbf{Esercizio 4}}
\vspace{0.5cm}
%\begin{center}
\footnotesize\noindent\fbox{
	\parbox{\textwidth}{
	 Si dia una maggiorazione del valore assoluto dell'errore relativo con cui \(x+y+z\) viene approssimato dall'approssimazione prodotta dal calcolatore, ossia \((x \oplus y) \oplus z\) (supporre che non ci siano problemi di overflow o di underflow). Ricavare l'analoga maggiorazione anche per \(x \oplus (y \oplus z)\) tenendo presente che \(x \oplus (y \oplus z) = (y \oplus z) \oplus x \).
	}
}\end{center}

Si considerino le seguenti operazioni di macchina:

\begin{itemize}

\item \((x \oplus y) \oplus z = fl\{ fl[ fl(x) + fl(y) ] + fl(z)\}\)
\item \((y \oplus z) \oplus x = fl\{ fl[ fl(y) + fl(z) ] + fl(x)\}\)

\end{itemize}

\noindent Ricordando che \(fl(a) = a + a\epsilon_a\) con \(\epsilon_a\) l'errore relativo commesso nella rappresentazione di \(a\) nel calcolatore, abbiamo che:

\[
\epsilon_{(x \oplus y) \oplus z } = \frac{(x \epsilon_x + y \epsilon_y) + z \epsilon_z}{(x+y)+z} \leq \frac{|(x+y)| + |z|}{|(x+y)+z|} \max\{\epsilon_x, \epsilon_y, \epsilon_z\}
\]

\noindent Analogamente, abbiamo:

\[
\epsilon_{(y \oplus z) \oplus x } \leq \frac{ |(y+z)| + |x| }{|(y+z) + x|}\max\{\epsilon_x, \epsilon_y, \epsilon_z\}
\]

\vspace{1cm}
\subsection{\textbf{Esercizio 5}}
\vspace{0.5cm}
%\begin{center}
\footnotesize\noindent\fbox{
	\parbox{\textwidth}{
	 Eseguire le seguenti istruzioni in Matlab:
	 \[
	 x=0;\ count = 0;\
	 while\ x \neq 1,\ x=x+delta, \ count = count +1, \ end
	 \]
	 dapprima ponendo \(delta = 1/16\) e poi ponendo \(delta = 1/20\).\\ Commentare i risultati ottenuti e in particolare il non funzionamento nel secondo caso.
	}
}\end{center}


\lstinputlisting[language=Matlab]{cap1/es5_1-16.m}

\begin{tabular}{l*{6}{c}}
 x  &  count \\
\hline
 0.0625 & 1 \\
 0.1250 & 2 \\
 0.1875 & 3 \\
 0.2500 & 4 \\
 0.3125 & 5 \\
 0.3750 & 6 \\
 0.4375 & 7 \\
 0.5000 & 8 \\
 0.5625 & 9 \\
 0.6250 & 10 \\
 0.6875 & 11 \\
 0.7500 & 12 \\
 0.8125 & 13 \\
 0.8750 & 14 \\
 0.9375 & 15 \\
      1 & 16 \\
\end{tabular} \\

\noindent Il ciclo \(while\) esegue esattamente 16 iterazioni prima di fermarsi sulla condizione \(x \neq 1\), che viene meno nell'ultimo passo. Questo avviene ovviamente in accordo all'incremento di \(x\) pari a \(\frac{1}{16}\) ad ogni iterazione, dato che la rappresentazione binaria di del numero decimale \(\frac{1}{16} = 0.0625\) risulta esatta.
\\
\\
\noindent Eseguendo invece il seguente codice Matlab, il ciclo non termina.

\lstinputlisting[language=Matlab]{cap1/es5_1-20.m}

\begin{tabular}{l*{6}{c}}
 x  &  count \\
\hline
 0.050000000000000 & 1 \\
 0.100000000000000 & 2 \\
 0.150000000000000 & 3 \\
 0.200000000000000 & 4 \\
 0.250000000000000 & 5 \\
 0.300000000000000 & 6 \\
 0.350000000000000 & 7 \\
 \ldots & \ldots \\
 1.000000000000000 & 20 \\
 \ldots & \ldots \\
\end{tabular} \\

Dato che il valore di \(delta\) questa volta risulta uguale a \(\frac{1}{20} = 0.05\), la sua rappresentazione nel calcolatore non risulta esatta, quindi deve essere approssimato. L'errore di approssimazione della variabile \(delta\) si accumula ad ogni iterazione nella variabile \(x\). Essendo la condizione del ciclo \(while\) vincolata al raggiungimento del numero esatto \(1\), non viene mai verificata ed il ciclo continua indefinitamente.



\newpage
\section{\textbf{Capitolo 3}}
\vspace{0.8cm}
\subsection{\textbf{Esercizio 1}}
\begin{center}
\footnotesize\noindent\fbox{
	\parbox{\textwidth}{
	Scrivere una function Matlab per la risoluzione di un sistema lineare con matrice dei coefficienti triangolare inferiore a diagonale unitaria. Inserire un esempio di utilizzo.
	}
}\end{center}

\noindent Si veda il seguente codice Matlab:

\lstinputlisting[language=Matlab]{cap3/es1.m}

\noindent La function realizzata risolve un sistema %eccetera eccetera

\vspace{1cm}
\subsection{\textbf{Esercizio 2}}
\begin{center}
\footnotesize\noindent\fbox{
	\parbox{\textwidth}{
	Utilizzare l'Algoritmo 3.6 del libro per stabilire sele seguenti matrici sono sdp o no:

\[
A_1 = \begin{pmatrix}1 & -1 & 2 & 2 \\ -1&5&-14&2\\ 2&-14&42&2\\2&2&2&65 \end{pmatrix},
A_2 = \begin{pmatrix}1 & -1 & 2 & 2 \\ -1&6&-17&3\\ 2&-17&48&-16\\2&3&-16&4 \end{pmatrix}
\]
	}
}\end{center}

\noindent Il seguente codice Matlab implementa una versione dell'algoritmo 3.6 del libro e la utilizza per rispondere a quanto richeisto. Nel merito, la matrice \(A_1\) \'e sdp, mentre la matrice \(A_2\) non lo \'e.

\lstinputlisting[language=Matlab]{cap3/3_2.m}

\vspace{1cm}
\subsection{\textbf{Esercizio 3}}
\begin{center}
\footnotesize\noindent\fbox{
	\parbox{\textwidth}{
	Scrivere una function Matlab che, avendo in ingresso un vettore \textbf{b} contenente i termini noti del sistema lineare \(Ax = b\) con \textit{A} sdp e l'output dell'Algoritmo 3.6 del libro (matrice \(A\) riscritta nella porzione triangolare inferiore con i fattori \(L\) e \(D\) della fattorizzazione \(LDL^T\) di \(A\)), ne calcoli efficientemente la soluzione.
	}
}\end{center}

\noindent Il seguente codice Matlab soddisfa la richiesta:
COMMETI SULL'EFFICIENZA

\lstinputlisting[language=Matlab]{cap3/3_3.m}

\vspace{1cm}
\subsection{\textbf{Esercizio 4}}
\begin{center}
\footnotesize\noindent\fbox{
	\parbox{\textwidth}{
	Scrivere una function Matlab che, avendo in ingresso un vettore \textbf{b} contenente i termini noti del sistema lineare \(Ax = b\)  e l'output dell'Algoritmo 3.7 del libro (matrice \(A\) riscritta con la fattorizzazione LU con pivoting parziale e il vettore \textbf{p} delle permutazioni), ne calcoli efficientemente la soluzione.
	}
}\end{center}

\noindent Il seguente codice Matlab soddisfa la richiesta:
COMMETI SULL'EFFICIENZA

\lstinputlisting[language=Matlab]{cap3/3_4.m}

\vspace{1cm}
\subsection{\textbf{Esercizio 5}}
\begin{center}
\footnotesize\noindent\fbox{
	\parbox{\textwidth}{
	Inserire alcuni esempi di utilizzo delle due function implementate per i punti 3 e 4, scegliendo per ciascuno di essi un vettore \(\overline{x}\) e ponendo \(b = A\overline{x}\). Riportare \(\overline{x}\) e la soluzione \(x\) da essi prodotta. Costruire anche una tabella in cui, per ogni esempio considerato, si riportano il numero di condizionamento di A in norma 2 (usare \lstinline[language=Matlab]{cond} di Matlab) e le quantit\'a \(||r||/||b||\) e \(||x-\overline{x}||/||\overline{x}||\).
	}
}\end{center}

\noindent Si presenta di seguito un esempio per la function dell'Esercizio 3 (risoluzione di sistemi con scomposizione \(LDL^T\)) ed un esempio per la function dell'Esercizio 4 (scomposizione LU con pivoting parziale). La matrice scelta per entrambi gli esempi \'e la seguente:
\[
A = \begin{pmatrix} 1 & -1 & 2 & 2 \\ -1 & 5 & -14 & 2\\ 2 & -14 & 42 & 2\\ 2 & 2 & 2 & 65 \end{pmatrix}
\]

\noindent I vettori scelti sono, rispettivamente per gli Esercizi 3 e 4:

\[
\overline{x}_3 = \begin{bmatrix} 3.1416 \\ 2.7183 \\ 1.4142 \\ 1.7320 \end{bmatrix},
\overline{x}_4 = \begin{bmatrix} 2.2360 \\ 2.6457 \\ 3.1416 \\ 3.3166 \end{bmatrix}
\]

\noindent Ponendo \(b=A\overline{x}\) si ha quindi:
\\
\[
b_3 = \begin{bmatrix} 6.7157 \\ -5.8849 \\ 31.0874 \\ 127.1282 \end{bmatrix},
b_4 = \begin{bmatrix} 12.5067 \\ -26.3567 \\ 106.0126 \\ 231.6256 \end{bmatrix}
\]

\noindent Lanciando le opportune function per la preparazione delle matrici e per la risoluzione del sistema (come mostrato nel dettaglio nel codice Matlab in coda a questa spiegazione), vengono prodotti i seguenti vettori soluzione:

\[
x_3 = \begin{bmatrix} 3.141600000000039 \\ 2.718300000000078 \\ 1.41420000000002 \\ 1.73200000000000 \end{bmatrix},
x_4 = \begin{bmatrix} 2.23599999999992 \\ 2.64569999999982 \\ 3.14159999999994 \\ 3.31660000000001 \end{bmatrix}
\]
\\
\noindent A seguire la tabella richiesta, che mostra il condizionamento della matrice restituita dagli algoritmi di fattorizzazione ed i confronti fra le quantit\'a richieste, analoghe ad errori relativi sui dati di ingresso (\(||r||/||b||\)) e sul risultato (\(||x-\overline{x}||/||\overline{x}||\)).
\\

\noindent\begin{tabular}{l*{20}{c}}
function & \vline& tipo & \vline& \(\overline{x}\) & \vline& \(cond(A, 2)\) & \vline& \(||r||/||b||\) & \vline& \(||x-\overline{x}||/||\overline{x}||\) \\
\hline
sol\_es\_3 & \vline& \(LDL^T\)    & \vline& \(\overline{x}_3\) & \vline& \(3.6158 \times 10^3\)    & \vline& \(1.0854 \times 10^{-16}\)	& \vline& \(1.9164 \times 10^{-14}\)   \\
sol\_es\_4 & \vline& \(LU\) pivot & \vline& \(\overline{x}_4\) & \vline& 319.10                    & \vline& \(1.2020 \times 10^{-16}\)	& \vline& \(3.5848 \times 10^{-14}\) 	 \\

\end{tabular}
\\
\\

\noindent Il codice Matlab usato per realizzare i precedenti esempi \'e il seguente. Le function riferite nel codice per le quali non \'e riportata l'implementazione sono quelle impiegate per risolvere gli esercizi precedenti.
\lstinputlisting[language=Matlab]{cap3/3_5.m}

\vspace{1cm}
\subsection{\textbf{Esercizio 6}}
\begin{center}
\footnotesize\noindent\fbox{
	\parbox{\textwidth}{
	Sia \(A = \begin{pmatrix} \epsilon & 1 \\ 1 & 1 \end{pmatrix}\) con \(\epsilon=10^{-13}\).  Definire L triangolare inferiore a diagonale unitaria e U diagonale superiore in modo che il prodotto LU sia la fattorizzazione LU di A e, posto \(b=Ae\) con \(e=(1,1)^{T}\), confrontare l'accuratezza della soluzione che si ottiene usando il comando \(U \ (L \ b)\) (Gauss senza pivoting) e il comando \(A \ b\) (Gauss con pivoting).
	}
}\end{center}

\vspace{1cm}
\subsection{\textbf{Esercizio 7}}
\begin{center}
\footnotesize\noindent\fbox{
	\parbox{\textwidth}{
Scrivere una function Matlab specifica per la risoluzione di un sistema lineare con matrice dei coefficienti \(A \in R^{n \times n}\) bidiagonale inferiore a diagonale unitaria di Toeplitz, specificabile con uno scalare \(\alpha\). Sperimentarne e commentarne le prestazioni (considerare il numero di condizionamento della matrice) nel caso in cui \(n=12\) e \(\alpha=100\) ponendo dapprima \(b=(1, 101, \ldots, 101)^T\) (soluzione esatta \(\overline{x}=(1,\ldots ,1)^T\)) e quindi \(b=0.1 * (1, 101, \ldots, 101)^T\) (soluzione esatta \(\overline{x}=(0.1,\ldots ,0.1)^T\)).
	}
}\end{center}

\noindent Ricordando che le matrici bidiagonali inferiori a diagonale unitaria di Toeplitz sono le matrici \(A \in R^{n \times n}\) del tipo \[\begin{pmatrix} 1 & 0 & 0 & \cdots & 0 \\ \alpha & 1 & 0 & \cdots & 0 \\ 0 & \alpha & 1 & \cdots & 0 \\ \vdots & \vdots & \vdots & \vdots & \vdots \\ 0 & \cdots & 0 & \alpha & 1\end{pmatrix}\]

\noindent \'E stato scritto il seguente codice Matlab per realizzare quanto chiesto:
\lstinputlisting[language=Matlab]{cap3/3_7.m}

\noindent Il condizionamento della matrice creata con le regole specificate risulta essere infinito, mentre riguardo al confronto delle soluzioni esatte con quelle ricavate dalla function implementata abbiamo quanto segue:
\[
b_1 = \begin{bmatrix}1\\101\\\vdots \\101 \end{bmatrix} \quad \overline{x}_1=\begin{bmatrix}1\\\vdots \\1 \end{bmatrix}^T = x_1
\]
\[
b_2 = 0.1 \times \begin{bmatrix}1\\101\\\vdots \\101 \end{bmatrix} \quad \overline{x}_2=\begin{bmatrix}0.1\\\vdots \\0.1 \end{bmatrix}^T \quad x_2=\begin{bmatrix}0.1000\\ \vdots \\ 0.1014 \\ -0.0407 \\ 14.1702 \\ -1.4069 \times 10^3 \\ 1.4070 \times 10^5 \end{bmatrix}^T
\]

AGGIUNGERE CONSIDERAZONI SULL'OSCILLAZIONE

%...
\vspace{1cm}
\subsection{\textbf{Esercizio 10}}
\begin{center}
\footnotesize\noindent\fbox{
	\parbox{\textwidth}{
	Scrivere una function Matlab che realizza il metodo di Newton per un sistema nonlineare (prevedere un numero massimo di iterazioni e utilizzare il criteri di arresto basato sull'incremento in norma euclidea). Utilizzare la function costruita al punto 4 per la risoluzione del sistema lineare ad ogni iterazione.
	}
}\end{center}

\noindent Il seguente codice Matlab mostra l'implementazione della funzione richiesta. \'E stato necessario usare, oltre alla function costruita nell'Esercizio 4, anche l'implementazione dell'algoritmo 3.7 mostrata nell'Eserczio 5.
\\
\lstinputlisting[language=Matlab]{cap3/3_10.m}



\newpage
\section{\textbf{Capitolo 4}}
\vspace{0.8cm}
\subsection{\textbf{Esercizio 1}}
\begin{center}
\footnotesize\noindent\fbox{
	\parbox{\textwidth}{
	Scrivere una function Matlab che implementi il calcolo del polinomio interpolante di grado \textit{n} in forma di Lagrange. \\ \\La forma della function deve essere del tipo: \lstinline[language=Matlab]{y = lagrange( xi, fi, x)}
	}
}\end{center}

\newpage
\subsection{\textbf{Esercizio 2}}
\begin{center}
\footnotesize\noindent\fbox{
	\parbox{\textwidth}{
	Scrivere una function Matlab che implementi il calcolo del polinomio interpolante di grado \textit{n} in forma di Newton. \\ \\La forma della function deve essere del tipo: \lstinline[language=Matlab]{y = newton( xi, fi, x)}
	}
}\end{center}

\noindent Il seguente codice Matlab implementa la function richiesta. Si noti che, analogamente a quanto fatto per l'esercizio precedente, \'e stata realizzata una sottofunzione che implementa il calcolo delle differenze divise, al fine di rendere il codice pi\'u chiaro. \\ \\
\noindent \'E stata anche in questo caso calcolata un'interpolazione di esempio, per verificare la correttezza di quanto fatto.

\lstinputlisting[language=Matlab]{cap4/4_2.m}

\newpage
\subsection{\textbf{Esercizio 3}}
\begin{center}
\footnotesize\noindent\fbox{
	\parbox{\textwidth}{
	Scrivere una function Matlab che implementi il calcolo del polinomio interpolante di Hermite. \\ \\La forma della function deve essere del tipo: \lstinline[language=Matlab]{y = hermite( xi, fi, f1i, x)}
	}
}\end{center}

\newpage
\subsection{\textbf{Esercizio 4}}
\begin{center}
\footnotesize\noindent\fbox{
	\parbox{\textwidth}{
	Utilizzare le functions degli esercizi precedenti per disegnare l'approssimazione della funzione \(\sin(x)\) nell'intervallo \([0, 2\pi]\), utilizzando le ascisse di interpolazione \(x_i=i\pi\), \(i=0,1,2\).
	}
}\end{center}

\newpage


\newpage
\section{\textbf{Capitolo 5}}
\vspace{0.8cm}
\subsection{\textbf{Esercizio 1}}
\begin{center}
\footnotesize\noindent\fbox{
	\parbox{\textwidth}{
	  Scrivere una function Matlab che implementi la formula composita dei trapezi su \(n+1\) ascisse equidistanti nell'intervallo \([a, b]\) relativamente alla funzione implementata da \lstinline[language=Matlab]{fun(x)}. \\ \\ La function deve essere del tipo \lstinline[language=Matlab]{If = trapcomp(a, b, fun, tol)}.
}
}\end{center}

\noindent Il seguente codice Matlab implementa la function richiesta. Non sono stati scritti test, in quanto la function verr\'a utilizzata nell'Esercizio 5, seppure in versione leggermente modificata. \\

\lstinputlisting[language=Matlab]{cap5/5_1.m}


\newpage
\subsection{\textbf{Esercizio 2}}
\input{cap5/5_2.tex}
\newpage
\subsection{\textbf{Esercizio 3}}
\input{cap5/5_3.tex}
\newpage
\subsection{\textbf{Esercizio 4}}
\begin{center}
\footnotesize\noindent\fbox{
	\parbox{\textwidth}{
	  Scrivere una function Matlab che implementi la formula composita di Simpson adattativa nell'intervallo \([a, b]\) relativamente alla funzione implementata da \lstinline[language=Matlab]{fun(x)} e con tolleranza \lstinline[language=Matlab]{tol}. \\ \\ La function deve essere del tipo \lstinline[language=Matlab]{If = simpad(a, b, fun, tol)}.
}
}\end{center}

\noindent Il seguente codice Matlab implementa la function richiesta. \\

\lstinputlisting[language=Matlab]{cap5/5_4.m}

\newpage
\subsection{\textbf{Esercizio 5}}
\begin{center}
\footnotesize\noindent\fbox{
	\parbox{\textwidth}{
Calcolare quante valutazioni di funzione sono necessarie per ottenere una approssimazione di
\[I(f) = \int_0^1 \exp(-10^6 x) dx \]
che vale \(10^-6\) in doppia precisione IEEE, con una tolleranza \(10^-9\), utilizzando le functions dei precedenti esercizi. Argomentare quantitativamente la risposta.
} }
\end{center}

\noindent Utilizzando versioni leggermente modificate delle function realizzate peri precedenti esercizi, sono stati calcolati i segueni valori:

\begin{itemize}

  \item \textbf{formula composita dei trapezi con \(n+1\) ascisse equidistanti} \\ con \(n = 10^7\), \(10^7 + 1\) valutazioni di \(\exp(-10^6 x)\), errore pari a \(8.3319 \times 10^{-10}\)
  \item \textbf{formula composita di Simpson su \(2n+1\) ascisse equidistanti } \\ con \(n = 2 \times 10^6\), \(2 \times 10^6 + 2\) valutazioni di \(\exp(-10^6 x)\), errore pari a \(3.3715 \times 10^{-10}\)
  \item \textbf{formula dei trapezi adattativa} \\ \(77823\) valutazioni di \(\exp(-10^6 x)\), errore pari a \(1.1253 \times 10^{-14}\)
  \item \textbf{formula di Simpson adattativa} \\ \(1038\) valutazioni di \(\exp(-10^6 x)\), errore pari a \(1.6470 \times 10^{-14}\)

\end{itemize}

\noindent Come si pu\'o vedere dai risultati, la scelta di ascisse equispaziate si rivela inaedeguata per una funzione come quella presa in esame, che presenta una rapida variazione di valore in una porzione dell'intervallo molto ristretta. Infatti, si riesce a catturare efficacemente questa variazione --- e quindi a raggiungere l'approssimazione richiesta sul risultato dell'integrale definito --- soltanto scegliendo di utilizzare un numero elevatissimo di punti, con conseguente bisogno di valutare moltissime volte la funzione.\\

\noindent Le formule adattive invece performano molto meglio, perch\'e individuano i nodi della partizione in base al comportamento locale della funzione, permettendo quindi di minimizzare l'errore e, di conseguenza, le chiamate ricorsive necessarie al raggiungimento della soglia di tolleranza prestabilita.\\

\noindent Il codice Matlab utilizzato per realizzare quanto descritto sopra \'e il seguente: \\

\lstinputlisting[language=Matlab]{cap5/5_5.m}

\newpage



\end{document}
