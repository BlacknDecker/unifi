\documentclass[a4paper]{article}

\usepackage{hyperref}
\usepackage{xcolor}
\usepackage{graphicx}
\usepackage[english]{babel}
\usepackage[T1]{fontenc}
\usepackage{url}
\usepackage{import}
\usepackage{multirow}
\usepackage{color}
\usepackage{fancyhdr}
\usepackage{amssymb}
\usepackage{tabu}
\usepackage{mathtools}
\usepackage[margin=2.5cm]{geometry}
\usepackage{listings}
\usepackage{titling}
\usepackage[utf8x]{inputenc}
\usepackage[numbered,framed]{matlab-prettifier}

\let\ph\mlplaceholder% shorter macro
\lstMakeShortInline"

\newcommand{\bnb}{\begin{nobreak}}
\newcommand{\enb}{\end{nobreak}}

\lstset{
  style              = Matlab-editor,
  basicstyle         = \mlttfamily,
  escapechar         = ",
  mlshowsectionrules = true,
}

\pretitle{%
  \begin{center}
  \huge
  \includegraphics[width=\textwidth/4]{logo_unifi.png}\\[\bigskipamount]
}
\posttitle{\end{center}}
\date{}

\begin{document}

\title{
	\vspace{3cm}
	\textbf{Calcolo Numerico}
	\\
	a.a. 2017/2018
	\vspace{1.5cm}
}

\author{
	Simone \textbf{Cipriani}, \texttt{5951907} --- \href{mailto:sim.cipr@gmail.com}{\textit{sim.cipr@gmail.com}}
	\vspace{1cm}
	\\
	Repository git del progetto hostata su GitHub:
	\\
	\url{https://github.com/gesucca/cn-class-project.git}
}

\maketitle
\newpage
\tableofcontents


\newpage
\section{\textbf{Capitolo 1}}
\subsection{\textbf{Esercizio 1.1}}
\vspace{0.5cm}
Dato \(\overline{x} = 2.7182\) e dato \(x = e = 2.7182818284\ldots \)

\begin{itemize}

\item errore assoluto: \(\Delta x = \overline{x} - x = 2.7182 - e \approx - 818 \times 10^{-7}\)
\item errore relativo: \(\epsilon_x = \frac{\Delta x}{x} = \frac{- 818 \times 10^{-7}}{e} = - 3 \times 10^{-5}\)

\end{itemize}

\noindent Si verfica adesso che:

\begin{itemize}
\item \(\overline{x} = 2.7182\) ha 4 cifre decimali
\item \(- \log_{10}|\epsilon_x| \approx 4.5 \)
\end{itemize}



\newpage
\section{\textbf{Capitolo 2}}
%\input{cap_2/cap_2.tex}

\newpage
\section{\textbf{Capitolo 3}}
%\input{cap_3/cap_3.tex}

\newpage
\section{\textbf{Capitolo 4}}
%\input{cap_4/cap_4.tex}

\newpage
\section{\textbf{Capitoli 5/6}}
%\vspace{0.8cm}
\subsection{\textbf{Esercizio 1}}
\begin{center}
\footnotesize\noindent\fbox{
	\parbox{\textwidth}{
	  Scrivere una function Matlab che implementi la formula composita dei trapezi su \(n+1\) ascisse equidistanti nell'intervallo \([a, b]\) relativamente alla funzione implementata da \lstinline[language=Matlab]{fun(x)}. \\ \\ La function deve essere del tipo \lstinline[language=Matlab]{If = trapcomp(a, b, fun, tol)}.
}
}\end{center}

\noindent Il seguente codice Matlab implementa la function richiesta. Non sono stati scritti test, in quanto la function verr\'a utilizzata nell'Esercizio 5, seppure in versione leggermente modificata. \\

\lstinputlisting[language=Matlab]{cap5/5_1.m}


\newpage
\subsection{\textbf{Esercizio 2}}
\input{cap5/5_2.tex}
\newpage
\subsection{\textbf{Esercizio 3}}
\input{cap5/5_3.tex}
\newpage
\subsection{\textbf{Esercizio 4}}
\begin{center}
\footnotesize\noindent\fbox{
	\parbox{\textwidth}{
	  Scrivere una function Matlab che implementi la formula composita di Simpson adattativa nell'intervallo \([a, b]\) relativamente alla funzione implementata da \lstinline[language=Matlab]{fun(x)} e con tolleranza \lstinline[language=Matlab]{tol}. \\ \\ La function deve essere del tipo \lstinline[language=Matlab]{If = simpad(a, b, fun, tol)}.
}
}\end{center}

\noindent Il seguente codice Matlab implementa la function richiesta. \\

\lstinputlisting[language=Matlab]{cap5/5_4.m}

\newpage
\subsection{\textbf{Esercizio 5}}
\begin{center}
\footnotesize\noindent\fbox{
	\parbox{\textwidth}{
Calcolare quante valutazioni di funzione sono necessarie per ottenere una approssimazione di
\[I(f) = \int_0^1 \exp(-10^6 x) dx \]
che vale \(10^-6\) in doppia precisione IEEE, con una tolleranza \(10^-9\), utilizzando le functions dei precedenti esercizi. Argomentare quantitativamente la risposta.
} }
\end{center}

\noindent Utilizzando versioni leggermente modificate delle function realizzate peri precedenti esercizi, sono stati calcolati i segueni valori:

\begin{itemize}

  \item \textbf{formula composita dei trapezi con \(n+1\) ascisse equidistanti} \\ con \(n = 10^7\), \(10^7 + 1\) valutazioni di \(\exp(-10^6 x)\), errore pari a \(8.3319 \times 10^{-10}\)
  \item \textbf{formula composita di Simpson su \(2n+1\) ascisse equidistanti } \\ con \(n = 2 \times 10^6\), \(2 \times 10^6 + 2\) valutazioni di \(\exp(-10^6 x)\), errore pari a \(3.3715 \times 10^{-10}\)
  \item \textbf{formula dei trapezi adattativa} \\ \(77823\) valutazioni di \(\exp(-10^6 x)\), errore pari a \(1.1253 \times 10^{-14}\)
  \item \textbf{formula di Simpson adattativa} \\ \(1038\) valutazioni di \(\exp(-10^6 x)\), errore pari a \(1.6470 \times 10^{-14}\)

\end{itemize}

\noindent Come si pu\'o vedere dai risultati, la scelta di ascisse equispaziate si rivela inaedeguata per una funzione come quella presa in esame, che presenta una rapida variazione di valore in una porzione dell'intervallo molto ristretta. Infatti, si riesce a catturare efficacemente questa variazione --- e quindi a raggiungere l'approssimazione richiesta sul risultato dell'integrale definito --- soltanto scegliendo di utilizzare un numero elevatissimo di punti, con conseguente bisogno di valutare moltissime volte la funzione.\\

\noindent Le formule adattive invece performano molto meglio, perch\'e individuano i nodi della partizione in base al comportamento locale della funzione, permettendo quindi di minimizzare l'errore e, di conseguenza, le chiamate ricorsive necessarie al raggiungimento della soglia di tolleranza prestabilita.\\

\noindent Il codice Matlab utilizzato per realizzare quanto descritto sopra \'e il seguente: \\

\lstinputlisting[language=Matlab]{cap5/5_5.m}

\newpage


\newpage
\pagenumbering{roman}
\section{\textbf{Figure}}
%\input{figure.tex}


\end{document}
