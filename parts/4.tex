\section{Classi di Complessità}

\subsection{Classi P e NP}

\begin{defn}[Classe P]
	Un linguaggio $L$ è decidibile in tempo \textbf{polinomiale} quando esiste una MdT $M$ \textbf{deterministica} che lo accetta con $tc_M(n) = O(n^r)$ per un qualche $r \in N$. \\ Quando questo si verifica, si dice che $L \in P$.
\end{defn}

\vspace{0.25cm}

\begin{defn}[Classe NP]
	Un linguaggio si dice decidibile in tempo \textbf{polinomiale non deterministico} quando esiste una MdT $M$ \textbf{non deterministica} che lo accetta con $tc_M(n) = O(n^r)$ per un qualche $r \in N$. \\ In questo caso si dice che $L \in NP$.
\end{defn}

Osserviamo che $P \subset NP$. \textit{Vale $P = NP$? Oppure $P \neq NP$?} Il problema è aperto. \\

Un \textbf{esempio di problema NP} è il problema del circuito hamiltoniano. Definiamo:

\begin{itemize}
	\item $G = (V, E)$ un grafo orientato, con $|V| = n$
	\item $\Gamma = (v_1, \ldots, v_n)$ il circuito hamiltoniano di $G$
		\subitem una permutazione di vertici distinti tale che $(v_i, v_i+1) \in E$ per ogni $i \leq n-1$
\end{itemize}

\noindent\textbf{Problema del circuito hamiltoniano}: \textit{dato G, determinare se ha un circuito hamiltoniano, cioè se esiste un cammmino che tocca tutti i vertici una volta sola.}

Tale problema è decidibile. Se per risolverlo ci avvaliamo di una MdT deterministica, la sua complessità temporale nel caso peggiore è esponenziale: occorre generare tutte le possibili sequenze di vertici e provare a percorrerle, quindi con $n$ vertici si avranno $n^n-1$ possibili sequenze. Usando invece una MdT non deterministica, è possibile risolverlo con complessità temporale polinomiale. Perciò $HAM \in NP$.

\subsection{Classe NP-Difficile}

\begin{defn}[Classe NPH]
	Un linguaggio $L$ si dice $Np-Difficile$ (o $NP-hard$) quando per ogni $L' \in NP$ esiste una riduzione polinomiale da $L'$ a $L$. In altre parole, un linguaggio è $NP-Difficile$ quando ogni linguaggio in $NP$ può essere ridotto ad esso con una riduzione polinomiale.
\end{defn}

\subsection{NP-Completezza}

\begin{defn}[Classe NPC]
	Quando un linguaggio è sia $NP-Difficle$ che appartenente ad $NP$, si dice che è $NP-Completo$.
\end{defn}

Se trovassimo un linguaggio $NP-Completo$ che appartenga alla classe $P$, questo dimostrerebbe che $P = NP$. Di seguito una formalizzazione di questo concetto:

\begin{lemm}[Implicazione della NP-Completezza]
	Sia $Q \in P$ e anche $NP-Completo$, allora per ogni $L \in NP$ vale anche $L \in P$.
\end{lemm}

\begin{proof}
	Se $Q$ è un problema $NP-Completo$, per sua definizione ogni linguaggio nella classe $NP$ può essere ridotto a $Q$. Se $Q \in P$, ogni linguaggio della classe $NP$ può essere ridotto ad un linguaggio della classe $P$, sigificando che in realtà $P = NP$. \\ Prendiamo $L \in NP$ come il generico linguaggio in classe $NP$, e sia $f$ una riduzione polinomiale da $L$ al nostro linguaggio $NP-Completo$ $Q$, che esiste sicuramente per definizione di $Np-Completezza$. Dato un input $w$, calcolo $f(w)$ e valuto se $f(w) \in Q$ usando una MdT deterministica polinomiale che ho per ipotesi dato che $Q \in P$. Questo appena descritto è un algoritmo di decisione per $L$ con complessità temporale polinomiale, quindi ogni generico $L \in P$ e, conseguentemente, $P = NP$. \\
	In estrema sintesi, per ogni $L \in NP$, se ho un $Q \in P$ che è anche $NP-Completo$, per ogni stringa $w$ calcolo la riduzione $f(w)$ e valuto se appartiene a $Q$, tutto in tempo polinomiale. Quindi, in realtà, ogni $L \in NP$ sarebbe anche in $P$.

\end{proof}

\begin{lemm}[Implicazione di P=NP]
	Se fosse $P = NP$, ogni linguaggio $L \in NP$ tale che $L \neq \O$ e che $L^C \neq \O$ è $NP-Completo$.
\end{lemm}

\begin{proof}
	Il fatto che $L$ ed il suo complementare $L^C$ non siano vuoti implica che esistono due stringhe $a \in L$ e $b \in L^C$. Consideriamo il generico linguaggio $Q \in NP$ e la riduzione polinomiale $f$ da $Q$ ad $L$. Avremo che:
	$$
		f(w) = \begin{cases} a, & \mbox{se } w \in Q \\ b, & \mbox{se } w \in Q^C \end{cases}
	$$
	Quindi, se $\O \notin NPC$, la riduzione esiste per tutti i linguaggi in $NP$.
\end{proof}

\begin{lemm}[Implicazione della riducibilità da linguaggi NPC]
	$L \in NPC$, $f$ riduzione polinomiale da $L$ a $Q$, $Q \in NP-Hard$. \\
	In altre parole, se c'è una riduzione polinomiale da un linguaggio $Np-Completo$ ad un altro linguaggio, quest'ultimo è $NP-Difficile$.
\end{lemm}

\begin{proof}
	Ovvio, perché se $L$ è $NP-Completo$ ogni linguaggio $NP$ può essere ridotto ad $L$ e la riduzione $f$ da $L$ a $Q$ esiste per ipotesi. Quindi, ogni linguaggio $NP$ può essere ridotto a $Q$ tramite la composizione delle funzioni di riduzione verso $L$ con $f$.
\end{proof}
