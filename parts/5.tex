\section{Soddisfacibilità Booleana}

\subsection{Polinomi Booleani}

Dato un insieme finito di variabili booleane $x_1, \ldots, x_n$, si definisce \textit{insieme dei polinomi booleani} in $x_1, \ldots, x_n$ l'insieme $PB(x_1, \ldots, x_n)$ contenente:
\begin{itemize}
	\item le costanti booleane 0 e 1
	\item le variabili booleane $x_1, \ldots, x_n$
	\item altri polinomi booleani $\in PB(x_1, \ldots, x_n)$ legati da operazioni di congiunzione, disgiunzione o negazione
\end{itemize}

\subsection{CNF --- Forma Normale Congiuntiva}

Definiamo due concetti:
\begin{itemize}
	\item \textbf{letterale} è una variabile, in forma semplice o negata
	\item \textbf{clausola} è una disgiunzioe di letterali
\end{itemize}

\vspace{0.4cm}

\begin{defn}
	Un polinomio booleano è in \textbf{Forma Normale Congiuntiva} quando è una congiunzione di clausole.
\end{defn}

\subsection{SAT}

Dato un polinomio booleano $p \in PB(x_1, \ldots, x_n)$ in $CNF$, determinare se esiste una n-upla $(t_1, \ldots, t_n) \in \{0, 1\}^n$ tale da soddisfare $p$.

\vspace{0.4cm}

\begin{lemm}
	$SAT \in NP$.
\end{lemm}

\begin{proof}
	Si può descrivere una MdT deterministica a due nastri che risolve il problema SAT in tempo esponenziale rispetto al numero di variabili, oppure renderla non deterministica e risolvere il SAT in tempo polinomiale. Questo implica che il problema SAT sia un problema $NP$.
\end{proof}

\vspace{0.4cm}

\begin{lemm}
	$SAT$ è $NP-Difficile$.
\end{lemm}

\begin{proof}
	Sia $L$ un generico linguaggio $NP$. Evidentemente, esiste una MdT $M$ non deterministica che accetta $L$ in tempo polinomiale. Possiamo descrivere una riduzione polinomiale da $L$ a $L_{SAT}$, ovvero descrivere una MdT che trasforma l'input $w$ di $M$ in un polinomio booleano tale che $M$ accetta $w$ se e solo se questo polinomio è soddisfacibile. \\ In altre parole, ogni $L \in NP$ è riducibile a $L_{SAT}$ tramite una riduzione polinomiale $f$ tale che $w \in L$ se e solo se $f(w) \in L_{SAT}$.
\end{proof}

\vspace{0.4cm}

\begin{lemm}
	\textbf{Teorema di Cook}: $SAT$ è un problema $NP-Completo$.
\end{lemm}

\begin{proof}
	Segue dai fatti che $SAT$ è $NP-Difficile$ e che $SAT \in NP$. La vera parte difficile è la definizione dellla MdT che riduce il generico linguaggio $L \in NP$ in $L_SAT$, infatti è stata omessa.
\end{proof}

\subsection{3SAT}

Si parte dagli stessi presupposti del problema $SAT$ originale, con la condizione addizionale che il polinomio di cui decidere la soddisfacibilità è in \textbf{3CNF}, ovvero le sue clausole contengono \textit{esattamente} 3 letterali.

\begin{lemm}
  $3SAT \in NP$.
\end{lemm}
