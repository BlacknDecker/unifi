\section{Soddisfacibilità Booleana}

\subsection{Polinomi Booleani}

Dato un insieme finito di variabili booleane $x_1, \ldots, x_n$, si definisce \textit{insieme dei polinomi booleani} in $x_1, \ldots, x_n$ l'insieme $PB(x_1, \ldots, x_n)$ contenente:
\begin{itemize}
	\item le costanti booleane 0 e 1
	\item le variabili booleane $x_1, \ldots, x_n$
	\item altri polinomi booleani $\in PB(x_1, \ldots, x_n)$ legati da operazioni di congiunzione, disgiunzione o negazione
\end{itemize}

\subsection{CNF --- Forma Normale Congiuntiva}

Definiamo due concetti:
\begin{itemize}
	\item \textbf{letterale} è una variabile, in forma semplice o negata
	\item \textbf{clausola} è una disgiunzioe di letterali
\end{itemize}

\begin{defn}
	Un polinomio booleano è in \textbf{Forma Normale Congiuntiva} quando è una congiunzione di clausole.
\end{defn}

\subsection{SAT}

Dato un polinomio booleano $p \in PB(x_1, \ldots, x_n)$ in $CNF$, determinare se esiste una n-upla $(t_1, \ldots, t_n) \in \{0, 1\}^n$ tale da soddisfare $p$.

\begin{lemm}
	$SAT \in NP$.
\end{lemm}

\begin{proof}
	Si può descrivere una MdT deterministica a due nastri che risolve il problema SAT in tempo esponenziale rispetto al numero di variabili, oppure renderla non deterministica e risolvere il SAT in tempo polinomiale. Questo implica che il problema SAT sia un problema $NP$.
\end{proof}

\begin{lemm}
	$SAT$ è $NP-Difficile$.
\end{lemm}

\begin{proof}
	Sia $L$ un generico linguaggio $NP$. Evidentemente, esiste una MdT $M$ non deterministica che accetta $L$ in tempo polinomiale. Possiamo descrivere una riduzione polinomiale da $L$ a $L_{SAT}$, ovvero descrivere una MdT che trasforma l'input $w$ di $M$ in un polinomio booleano tale che $M$ accetta $w$ se e solo se questo polinomio è soddisfacibile.
\end{proof}

In altre parole, ogni $L \in NP$ è riducibile a $L_{SAT}$ tramite una riduzione polinomiale $f$ tale che $w \in L$ se e solo se $f(w) \in L_{SAT}$.

\begin{lemm}[\textbf{Teorema di Cook}]
	$SAT$ è un problema $NP-Completo$.
\end{lemm}

\begin{proof}
	Segue dai fatti che $SAT$ è $NP-Difficile$ e che $SAT \in NP$. La parte complicata della dimostrazione è la definizione dellla MdT che riduce il generico linguaggio $L \in NP$ in $L_{SAT}$, infatti è stata omessa.
\end{proof}

\subsection{3SAT}

Si parte dagli stessi presupposti del problema $SAT$ originale, con la condizione addizionale che il polinomio di cui decidere la soddisfacibilità è in \textbf{3CNF}, ovvero le sue clausole contengono \textit{esattamente} 3 letterali.

\begin{lemm}
  $3SAT \in NP$.
\end{lemm}

\begin{proof}
	Segue immediatamente dal fatto che $SAT$ sia un problema $NP$ (alla fine $SAT$ è solo una definizione più generale di $3SAT$).
\end{proof}

\begin{lemm}
	$3SAT$ è un problema $NP-Completo$.
\end{lemm}

\begin{proof}
	Per ridurre $SAT$ a $3SAT$, basta dimostrare che si può trasformare un polinomio booleano generico in uno equivalente le cui clausole abbiano esattamente 3 letterali. \\ \\
	Dato un polinomio $p \in PB(n)$ in $CNF$, sia $f : PB(n) \rightarrow PB(n)$ la nostra riduzione polinomiale da $SAT$ a $3SAT$: $f(p)$ sarà un polinomio $p' \in PB(n)$ in $3CNF$ tale che $p$ sia soddisfacibile se e solo se $p'$ lo è.  La descrizione della riduzione $f$ è piuttosto intricata, perciò la saltiamo del tutto.
\end{proof}

\begin{remark}
	$3SAT$ è il primo fra tutti i $k-SAT$ ad essere $NPC$. Spesso per dimostrare che un problema è $NPC$, si cerca una riduzione ad esso da $3SAT$ perché è il più semplice in $NPC$.
\end{remark}

\subsection{2SAT}

Si trova nella classe $P$, si può dimostrare riducendolo ad un problema sui grafi in $P$.
