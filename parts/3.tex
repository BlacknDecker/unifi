\section{Classi di Complessità}

\subsection{Macchine di Turing non deterministiche}

\begin{lemm}
	Sia $M$ una MdT non deterministica con grado di non determinismo $\sigma$ e con $tc_M(n) = f(n)$. Esiste una MdT non deterministica $M'$ equivalente ad $M$ tale che $tc_{M'}(n) = O(f(n)\sigma^{f(n)})$.
\end{lemm}

\begin{proof}
	Basta trasfrmare $M$ in una MdT deterministica: presa in input una stringa $w$ di lunghezza $n$, si genera una tupla di interi $(m_1, \ldots, m_{f(n)})$ con ogni $m_j \leq \sigma$ e si simula la computazione associata. Il costo nel caso peggiore di questo scenario è $f(n)\sigma^{f(n)}$.
\end{proof}

\subsection{Classi P e NP}

\begin{defn}
	Un linguaggio $L$ è decidibile in tempo \textbf{polinomiale} quando esiste una MdT $M$ \textbf{deterministica} che lo accetta con $tc_M(n) = O(n^r)$ per un qualche $r \in N$. Quando questo si verifica, si dice che $L \in P$.
\end{defn}

\begin{defn}
	Un linguaggio si dice decidibile in tempo \textbf{polinomiale non deterministico} quando esiste una MdT $M$ \textbf{non deterministica} che lo accetta con $tc_M(n) = O(n^r)$ per un qualche $r \in N$. In questo caso si dice che $L \in NP$.
\end{defn}

Osserviamo che $P \subset NP$. \textit{Vale $P = NP$? Oppure $P \neq NP$?} Il problema è aperto.

Un \textbf{esempio di problema NP} è il problema del circuito hamiltoniano. Definiamo:

\begin{itemize}
	\item $G = (V, E)$ un grafo orientato, con $|V| = n$
	\item $\Gamma = (v_1, \ldots, v_n)$ il circuito hamiltoniano di $G$
		\subitem una permutazione di vertici distinti tale che $(v_i, v_i+1) \in E$ per ogni $i \leq n-1$
\end{itemize}

\textbf{Problema del circuito hamiltoniano}: \textit{dato G, determinare se ha un circuito hamiltoniano.}

Tale problema è decidibile. Se per risolverlo ci avvaliamo di una MdT deterministica, la sua complessità tmeporale è esponenziale. Usando invece una MdT non deterministica, è possibile risolverlo con complessità tmeporale polinomiale. Perciò $HAM \in NP$.
