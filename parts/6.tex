\section{Altri problemi $NP-Completi$}

\subsection{Vertex Cover}

Sia $G = (V, E)$ un grafo non orientato. Si definisce \textit{vertex cover} --- o, in italiano, copertura tramite vertici --- di $G$ un sottoinsieme $C \subseteq V$ tale che per ogni $\{x, y\} \in E$, $x \text{ o } y \in C$.

In altre parole, il vertex cover è un insieme di vertici tale che ogni arco del grafico ne tocchi almeno uno.

\vspace{0.25cm}

\begin{defn}[Vertex Cover]
    Dato un grafo non orientato $G$ ed un intero $k$, determinare se $G$ ha un vertex cover di cardinalità $k$.
\end{defn}

\vspace{0.25cm}

\begin{lemm}
	$VC \in NP$.
\end{lemm}

\begin{proof}
	Per dimostrarlo basta descrivere una riduzione polinomiale da $3SAT$ a $VC$. Si tratta quindi di mappare un polinomio binario in $3CNF$ in un grafo del quale occorre verificare la presenza di un vertex cover. \\
    Prendiamo quindi $p \in 3CNF$ e definiamo come $G(p)$ il grafo associato a $p$ così composto:
    \begin{itemize}
        \item $V_1 = \{x_1, x_1' \ldots , x_n, x_n'\}$ vertici di $G(p)$ che rappresentano le variabili di $p$, sia in forma normale che negata
        \item $V_2 = \{u_{1,1}, u_{1,2}, u_{1,3}, \ldots , u_{n,1}, u_{n,2}, u_{n,3}\}$ vertici che rappresentano i letterali delle clausole di $p$
        \item $V = V_1 \bigcup V_2$ insieme di tutti i vertici di $G(p)$
        \item Un complesso sistema di specchi e leve$^{cit.}$ che consiste nel collegare fra loro con degli archi le coppie $x_i, x_i'$, le terne $u_{n,1}, u_{n,2}, u_{n,3}$ ed i vari $u_{i,j}$ alla variabile che rappresentano in $p$.
    \end{itemize}

    \begin{remark}
	    Il vertex cover di $G(p)$ è composto da almeno $n+2m$ vertici, con $n$ numero delle variabili e $m$ numero delle clausole.
    \end{remark}

    Ora, sia $t$ l'assegnamento che soddisfa $p$. Possiamo ricostruire il vertex cover $C \subseteq V$ di cardinalità $n+2m$ considerando l'insieme $C = \{x_i \text{variabile di } p | t(x_i)=1\} \bigcup \{\{x_i' \text{variabile di } p | t(x_i')=1\}$. Intanto abbiamo un insieme di $n$ elementi che copre i vertici rappresentanti le variabili. Per coprire anche i vertici $u_{j,s}$ che rappresentano i letterali, per ogni clausola $u_j$ vado a vedere quale è collegata al letterale tale che $t(u_{js}) = 1$ ed aggiungo a $C$le altre due. In questo modo si aggiungono altri $2m$ elementi, ed ho finalmente la cardinalità $n+2m$. \\
    Quella mostrata è una riduzione polinomiale da $3SAT$ a $VC$, in quanto $p \in PB(n)$ in $3CNF$ con $m$ clausole risulta soddisfacibile solo se $G(p)$ ha almeno un vertex cover di cardinalità $n+2m$. Ciò significa che $VC$ è un problema $NP-Difficile$, ed in quanto problema $NP$, è anche $NP-Completo$.
\end{proof}

\subsection{Clique}

Coming soon?
