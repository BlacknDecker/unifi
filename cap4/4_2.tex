\begin{center}
\footnotesize\noindent\fbox{
	\parbox{\textwidth}{
	Scrivere una function Matlab che implementi il calcolo del polinomio interpolante di grado \textit{n} in forma di Newton. \\ \\La forma della function deve essere del tipo: \lstinline[language=Matlab]{y = newton( xi, fi, x)}
	}
}\end{center}

\noindent Il seguente codice Matlab implementa la function richiesta. Si noti che, analogamente a quanto fatto per l'esercizio precedente, \'e stata realizzata una sottofunzione che implementa il calcolo delle differenze divise, al fine di rendere il codice pi\'u chiaro. \\ \\
\noindent \'E stata anche in questo caso calcolata un'interpolazione di esempio, per verificare la correttezza di quanto fatto.

\lstinputlisting[language=Matlab]{cap4/4_2.m}
