\begin{center}
\footnotesize\noindent\fbox{
	\parbox{\textwidth}{
	Utilizzare le function degli Esercizi 4.1 e 4.6 per graficare l'approssimazione della funzione di Runge sull'intervallo \([-6, 6]\), per \(n = 2, 4, 6, \ldots, 40\). \\ \\Stimare numericamente l'errore commesso in funzione del grado \textit{n} del polinomio interpolante.
	}
}\end{center}

\noindent Di seguito i grafici che mostrano i polinomi interpolanti di grado \textit{n} calcolati usando come punti di interpolazione quelli corrispondenti alle \textit{n} ascisse di Chebyshev. \\

\noindent\small\begin{tabular}{l*{5}{c}}
\hspace{3.5cm}\(n=2\) & \(n=4\) \\
\includegraphics[scale=0.5]{cap4/4_7/2.png} &  \includegraphics[scale=0.5]{cap4/4_7/4.png} \\

\hspace{3.5cm}\(n=6\)& \(n=8\) \\
\includegraphics[scale=0.5]{cap4/4_7/6.png} &  \includegraphics[scale=0.5]{cap4/4_7/8.png} \\

\hspace{3.5cm}\(n=10\) &  \(n=12\) \\
\includegraphics[scale=0.5]{cap4/4_7/10.png} &  \includegraphics[scale=0.5]{cap4/4_7/12.png} \\
\end{tabular} \\ \\

\small\begin{tabular}{l*{5}{c}}
\hspace{3.5cm}\(n=14\) &  \(n=16\) \\
\includegraphics[scale=0.5]{cap4/4_7/14.png} &  \includegraphics[scale=0.5]{cap4/4_7/16.png} \\

\hspace{3.5cm}\(n=18\) &  \(n=20\) \\
\includegraphics[scale=0.5]{cap4/4_7/18.png} &  \includegraphics[scale=0.5]{cap4/4_7/20.png} \\

\hspace{3.5cm}\(n=22\) &  \(n=24\) \\
\includegraphics[scale=0.5]{cap4/4_7/22.png} &  \includegraphics[scale=0.5]{cap4/4_7/24.png} \\

\hspace{3.5cm}\(n=26\) &  \(n=28\) \\
\includegraphics[scale=0.5]{cap4/4_7/26.png} &  \includegraphics[scale=0.5]{cap4/4_7/28.png} \\
\end{tabular}

\small\begin{tabular}{l*{5}{c}}
\hspace{3.5cm}\(n=30\) &  \(n=32\) \\
\includegraphics[scale=0.5]{cap4/4_7/30.png} &  \includegraphics[scale=0.5]{cap4/4_7/32.png} \\

\hspace{3.5cm}\(n=34\) &  \(n=36\) \\
\includegraphics[scale=0.5]{cap4/4_7/34.png} &  \includegraphics[scale=0.5]{cap4/4_7/36.png} \\

\hspace{3.5cm}\(n=38\) &  \(n=40\) \\
\includegraphics[scale=0.5]{cap4/4_7/38.png} &  \includegraphics[scale=0.5]{cap4/4_7/40.png} \\
\hspace{2.6cm}\(r(x) = \frac{1}{1+x^2}\): &  \\
\includegraphics[scale=0.5]{cap4/4_7/runge.png} & \\
\end{tabular}
\\

\noindent Il codice Matlab utilizzato per generare i precedenti grafici \'e il seguente.

\lstinputlisting[language=Matlab]{cap4/4_7.m}

\noindent Riguardo alla stima dell'errore commesso, ricordando che, per funzioni sufficientemente regolari:
\[
||e|| = \frac{||f^{(n+1)}||}{(n+1)!2^n}
\]
