\begin{center}
\footnotesize\noindent\fbox{
	\parbox{\textwidth}{
	Stimare, nel senso dei minimi quadrati, posizione, velocit\'a iniziale ed accelerazione relative ad un moto rettilineo uniformemente accelerato per cui sono note le seguenti misurazioni dele coppie \((tempo, spazio)\):\\
	\((1, 2.9) \quad (1, 3.1)\quad (2, 6.9) \quad (2, 7.1) \quad (3, 12.9) \quad (3, 13.1) \quad (4, 20.9) \quad (4, 21.1) \quad (5, 30.9) \quad (5, 31.1)\)
	}
}\end{center}

\noindent Si parla di moto rettilineo uniformemente accelerato, perci\'o fissando l'origine del sistema di riferimento spaziale in modo da avere \(x_0 = 0\), ricordiamo che:

\[
  x(t) = v_0t +  \frac{1}{1}at^2
\]

\noindent e che:

\[
  v = v_0 + at 
\]

\noindent La legge che descrive il fenomeno (in realt\'a ben noto, come appena mostrato) \'e di tipo polinomiale, e mette in relazione due grandezze sperimentali quali un intervallo di tempo ed uno spostamento. Si ricerca quindi il polinomio che meglio approssima le coppie di dati fornite, nel senso di nei minimi quadrati. Tale polinomio avr\'a la seguente forma:

\[
  y = \sum^m_{k=0}a_kx^k
\]

\noindent con il vettore dei coefficienti \textbf{a} tale da minimizzare la quantit\'a \(||\textbf{y}-\textbf{z}||^2_2 = \sum^n_{i=0}|y_i-z_i|^2\), ed il vettore \textbf{z} definito come segue:

\[
  \textbf{z} = \begin{pmatrix} \sum_{k=0}^m a_kx_0^k \\ \vdots \\  \sum_{k=0}^m a_kx_n^k \end{pmatrix} = V \textbf{a}
\]

\noindent Si tratta quindi di risolvere il sistema sovradeterminato \(V \textbf{a} = \textbf{y} \), con \(V\) matrice di tipo Vandermonde.

