\begin{center}
\footnotesize\noindent\fbox{
	\parbox{\textwidth}{
		Sia \(x=e \approx 2.7183 = \overline{x}\). Si calcoli il corrispondente errore relativo \(\epsilon_x\) e il numero di cifre significative \(k\) con cui \(\overline{x}\) approssima \(x\). Si verifichi che \(|\epsilon_x|\approx \frac{1}{2}10^{-k}\).
	}
}\end{center}

\noindent Dato \(\overline{x} = 2.7183\) e dato \(x = e = 2.718281828459\ldots \) si calcolano l'errore assoluto e l'errore relativo.

\begin{itemize}

\item errore assoluto: \(\Delta x = \overline{x} - x = 2.7183 - e \approx 1.82 \times 10^{-5}\)
\item errore relativo: \(\epsilon_x = \frac{\Delta x}{x} = \frac{ 1.82 \times 10^{-5}}{e} \approx 6.68 \times 10^{-6}\)

\end{itemize}

\noindent Si verfica adesso che:

\begin{itemize}
\item \(\overline{x} = 2.7183\) approssima \(e\) con \(k=5\) cifre significative
\item \( \frac{1}{2}10^{-k} = 0.5 \times 10^{-5} \approx 6.68 \times 10^{-6} = |\epsilon_x|\)
\end{itemize}
