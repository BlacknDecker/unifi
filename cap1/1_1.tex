Dato \(\overline{x} = 2.7182\) e dato \(x = e = 2.7182818284\ldots \)

\begin{itemize}

\item errore assoluto: \(\Delta x = \overline{x} - x = 2.7182 - e \approx - 818 \times 10^{-7}\)
\item errore relativo: \(\epsilon_x = \frac{\Delta x}{x} = \frac{- 818 \times 10^{-7}}{e} = - 3 \times 10^{-5}\)

\end{itemize}

\noindent Si verfica adesso che:

\begin{itemize}
\item \(\overline{x} = 2.7182\) ha 4 cifre decimali
\item \(- \log_{10}|\epsilon_x| \approx 4.5 \)
\end{itemize}
