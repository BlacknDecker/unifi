\begin{center}
\footnotesize\noindent\fbox{
	\parbox{\textwidth}{
	 Si dia una maggiorazione del valore assoluto dell'errore relativo con cui \(x+y+z\) viene approssimato dall'approssimazione prodotta dal calcolatore, ossia \((x \oplus y) \oplus z\) (supporre che non ci siano problemi di overflow o di underflow). Ricavare l'analoga maggiorazione anche per \(x \oplus (y \oplus z)\) tenendo presente che \(x \oplus (y \oplus z) = (y \oplus z) \oplus x \).
	}
}\end{center}

Si considerino le seguenti operazioni di macchina (senza consierare eventuali condizioni di overflow ed underflow):

\begin{itemize}

\item \((x \oplus y) \oplus z = fl\{ fl[ fl(x) + fl(y) ] + fl(z)\}\)
\item \((y \oplus z) \oplus x = fl\{ fl[ fl(y) + fl(z) ] + fl(x)\}\)

\end{itemize}

\noindent Ricordando che \(fl(a) = a + a\epsilon_a\) con \(\epsilon_a\) l'errore relativo commesso nella rappresentazione di \(a\) nel calcolatore, abbiamo che:

\[
\epsilon_{(x \oplus y) \oplus z } = \frac{(x \epsilon_x + y \epsilon_y) + z \epsilon_z}{(x+y)+z} \leq \frac{|(x+y)| + |z|}{|(x+y)+z|} \max\{\epsilon_x, \epsilon_y, \epsilon_z\}
\]

\noindent Analogamente, abbiamo:

\[
\epsilon_{(y \oplus z) \oplus x } \leq \frac{ |(y+z)| + |x| }{|(y+z) + x|}\max\{\epsilon_x, \epsilon_y, \epsilon_z\}
\]
