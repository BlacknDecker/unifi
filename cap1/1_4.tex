\begin{center}
\footnotesize\noindent\fbox{
	\parbox{\textwidth}{
	 Si dia una maggiorazione del valore assoluto dell'errore relativo con cui \(x+y+z\) viene approssimato dall'approssimazione prodotta dal calcolatore, ossia \((x \oplus y) \oplus z\) (supporre che non ci siano problemi di overflow o di underflow). Ricavare l'analoga maggiorazione anche per \(x \oplus (y \oplus z)\) tenendo presente che \(x \oplus (y \oplus z) = (y \oplus z) \oplus x \).
	}
}\end{center}

\noindent Si consideri la definizione dell'operazione somma in aritmetica di macchina e la si applichi al caso richiesto:

\begin{itemize}

\item \((x \oplus y) \oplus z = fl( fl( fl(x) + fl(y) ) + fl(z)) \)
\item \((y \oplus z) \oplus x = fl( fl( fl(y) + fl(z) ) + fl(x)) \)

\end{itemize}

\noindent Ricordando che \(fl(a) = a + a\epsilon_a = a (1 + \epsilon_a)\) e che \(|\epsilon_a| \leq u\) con \(u\) precisione di macchina, possiamo ricavare:

\[
|\epsilon_{(x \oplus y) \oplus z }| = \frac{|fl( fl( fl(x) + fl(y) ) + fl(z)) - (x + y + z)|}{|x+y+z|}
\]
\[=
\frac{|((x + x\epsilon_x + y + y\epsilon_y)(1+\epsilon_{x \oplus y}) + z + z\epsilon_z)(1 +\epsilon_{(x \oplus y) \oplus z }) - (x+y+z)|}{|x+y+z|}
\]
\[
=\frac{\left|\splitfrac{x\epsilon_x + y\epsilon_y + x\epsilon_{x \oplus y} + x\epsilon_x\epsilon_{x \oplus y} + y\epsilon_{x \oplus y} + y\epsilon_y\epsilon_{x \oplus y} + z\epsilon_z + x\epsilon_x\epsilon_{(x \oplus y) \oplus z} + y\epsilon_y\epsilon_{(x \oplus y) \oplus z}    }{     + x\epsilon_{x \oplus y}\epsilon_{(x \oplus y) \oplus z} + x\epsilon_x\epsilon_{x \oplus y}\epsilon_{(x \oplus y) \oplus z} + y\epsilon_{x \oplus y}\epsilon_{(x \oplus y) \oplus z} + y\epsilon_y\epsilon_{x \oplus y}\epsilon_{(x \oplus y) \oplus z} + z\epsilon_z\epsilon_{(x \oplus y) \oplus z}}\right|}{|x+y+z|}
\]
\[
\leq \frac{\left|\splitfrac{\max \{\epsilon_x, \epsilon_y, \epsilon_z\}(2x+2y+z) }{  \splitfrac{+ \max \{\epsilon_x, \epsilon_y, \epsilon_z, \epsilon_{x \oplus y}, \epsilon_{(x \oplus y) \oplus z}\} ^2 (3x+3y+z) }{+  \max \{\epsilon_x, \epsilon_y, \epsilon_z, \epsilon_{x \oplus y}, \epsilon_{(x \oplus y) \oplus z}\} ^3 (x+y)}}\right|}{|x+y+z|}
\]
\[
\leq \max \{\epsilon_x, \epsilon_y, \epsilon_z, \epsilon_{x \oplus y}, \epsilon_{(x \oplus y) \oplus z}\} ^ 3 \frac{|6x + 6y + 2z|}{|x+y+z|}
\]
\[
\leq u^3 \frac{|6x + 6y + 2z|}{|x+y+z|}
\]
\\
\noindent Svolgendo un procedimento del tutto analogo al precedente, nel caso di \((y \oplus z) \oplus x \) avremo:

\[
\epsilon_{(y \oplus z) \oplus x } \leq u^3 \frac{|6y + 6z + 2x|}{|x+y+z|}
\]
