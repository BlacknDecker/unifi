Si esegue il seguente codice in Matlab

\lstinputlisting[language=Matlab]{cap1/es5_1-16.m}

\begin{tabular}{l*{6}{c}}
 x  &  count \\
\hline
 0.0625 & 1 \\
 0.1250 & 2 \\
 0.1875 & 3 \\
 0.2500 & 4 \\
 0.3125 & 5 \\
 0.3750 & 6 \\
 0.4375 & 7 \\
 0.5000 & 8 \\
 0.5625 & 9 \\
 0.6250 & 10 \\
 0.6875 & 11 \\
 0.7500 & 12 \\
 0.8125 & 13 \\
 0.8750 & 14 \\
 0.9375 & 15 \\
      1 & 16 \\
\end{tabular} \\

\noindent TODO: bla bla bla spiegazioni

\noindent Eseguendo invece il seguente codice Matlab:

\lstinputlisting[language=Matlab]{cap1/es5_1-20.m}

Il valore di delta  uguale a \(\frac{1}{10}\). La rappresentazione binaria di questo numero per non  esatta.
Si tratta di una rappresentazione periodica e quindi, in decimale, sar circa 0.0999.\\
prendendo quindi $delta\approx0.9$ vedremo che per $i=10 \mbox{   }x\approx0.999$.Mentre, per $i=11$  $x\approx1,0989$.\\
Per questo motivo la condizione x=1 non si avverer mai e il programma non terminer.
