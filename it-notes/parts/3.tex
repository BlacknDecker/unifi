\section{Riducibilità Polinomiale}

\begin{defn}[Riducibilità tramite funzione di riduzione]
	Siano $L, Q$ due linguaggi su rispettivi alfabeti $\Sigma_1$ e $\Sigma_2$. $L$ è polinomialmente riducibile a Q se esiste una funzione $f : \Sigma_1 \rightarrow \Sigma_2$ computabile in tempo polinomiale e tale che per ogni $w \in \Sigma_1$, $w \in L$ se e solo se $f(w) \in Q$.
\end{defn}

In altre parole, $L$ è polinomialmente riducibile a Q quando esiste una funzione che trasforma le parole di $L$ in parole di $Q$ in tempo polinomiale. \\

\begin{remark}
	Le MdT che calcolano le funzioni di riduzione polinomiale sono tutte deterministiche.
\end{remark}

\vspace{0.25cm}

\begin{defn} [Implicazione della riducibilità]
	Se $L$ è polinomialmente riducibile a $Q$, e $Q \in P$, allora $L \in P$.
\end{defn}

\begin{proof}
	Data $w \in \Sigma_L^\ast$, calcolo $f(w) \in \Sigma_Q^\ast$ e valuto se appartiene a $Q$ oppure no. Supponiamo di avere le seguenti MdT: $M_f$ che calcola la riduzione polinomiale $f$ con $tc_{M_f}(n) = O(n^r)$, $M_Q$ che accetta Q con $tc_{M_Q}(n) = O(n^s)$, entrambe deterministiche per ipotesi. La MdT che accetta $L$ lancia in successione $M_f$ e $M_Q$, perciò $L \in P$.
\end{proof}
