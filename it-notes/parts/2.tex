\section{Speedup Lineare}

Partiamo dai seguenti presupposti:

\begin{itemize}
	\item $M$ una MdT a due nastri
	\item $\Sigma$ l'alfabeto finito di M
	\item $Q$ l'insieme dei possibili stati di M
\end{itemize}

Costruiamo una MdT $N$, sempre a due nastri, in cui ogni gruppo di 3 transazioni di $M$ viene simulato da un numero costante di transazioni di $N$. \\

L'\textbf{alfabeto} di $N$ quindi sarà: $ \Sigma \cup \ \{\#\} \cup \{ \text{triple di elementi di } \Sigma \cup \{\ast\} \cup \{?\} \} $ \\

Gli \textbf{stati} di $N$ saranno invece quartuple del tipo $(q_1, (t_1, t_2, t_3), (u_1, u_2, u_3), (i_1, i_2)$ dove:

\begin{itemize}
	\item $q \in Q$ è lo stato in cui si trova la versione simulata di $M$
	\item la tripla $(t_1, t_2, t_3)$ rappresenta le informazioni del nastro 1:
		\subitem $t_1$ è la tripla precedente a $t_2$
		\subitem $t_2$ è la tripla su cui si trova la testina
		\subitem $t_1$ è la tripla successiva a $t_2$
	\item la tripla $(u_1, u_2, u_3)$ rappresenta le informazioni analoghe relative al nastro 2
	\item $i_1, i_2 \in \{1, 2, 3\}$ indica su quale elemento delle triple $t_2, u_2$ si trova la testina di M
\end{itemize}

In questo modo gli stati della MdT $N$ contengono le informazioni necessarie per simulare le transazioni di $M$ a gruppetti di tre per volta. \\

\begin{lemm}[Speedup lineare]
	Sia $M$ una MdT a k nastri, con $k > 1$, che accetta $L$ con $tc_M(n) = f(n)$.
	Per ogni $c > 0$ esiste una MdT $N$ che accetta L con $tc_N(n) \leq c*f(n) + 4n + 5$.
\end{lemm}

\begin{proof}
	Sia $m \in N$. La simulazione di $M$ su $N$ richiede il seguente numero di transazioni:
	\begin{itemize}
		\item \textbf{Inizializzazione} --- $4n + 5$
			\subitem copiare e codificare in $m$-uple l'input dal nastro 1 al nastro 2 --- $2(n+1)$
			\subitem riportare la testina all'inizio dei nastri --- $2(n+1)$
			\subitem scrivere simbolo di fine nastro $\#$ --- 1
		\item \textbf{Simulazione} --- $8\frac{f(n)}{m}$
			\subitem lettura cella a sinistra --- 2
			\subitem lettura cella a sinistra --- 3
			\subitem tornare alla cella iniziale --- 1
			\subitem simulazione della $m$-upla --- al più 2 (spostamento e scrittura) \\ \\
		Tutto questo ripetuto per $\frac{f(n)}{m}$ volte.
	\end{itemize}
	In totale quindi il costo in transazioni dell'intera computazione è $8\frac{f(n)}{m} + 4n + 5$.
\end{proof}

Il fatto più rilevante è che si possono simulare $m$ transazioni di $M$ in 8 trnsazioni di N. Quindi, fissato un opportuno $c > 0$, basta scegliere $m > \frac{8}{c}$ per far sì che $N$ sia più veloce di $M$.
