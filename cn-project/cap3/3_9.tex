\begin{center}
\footnotesize\noindent\fbox{
	\parbox{\textwidth}{
Inserire due esempi diutilizzo della function implementata per il punto 8 e confrontare la soluzione ottenuta con quella fornita dal comando \(A \textbackslash b\).
	}
}\end{center}

\noindent Siano per il primo esempio:
\[
A_1 = \begin{pmatrix} 3 & 1 & 2 \\ 1 & 1& 3 \\ 1 & 3 & 4 \\ 2 & 1 & 2 \end{pmatrix} \quad b_1 = \begin{bmatrix} 4\\6\\4\\2 \end{bmatrix}
\]

\noindent La funzione implementata restituisce questa soluzione: \(x_1 = \begin{bmatrix} 0.128491620111731 \\ -2.229050279329608 \\ 2.603351955307263 \end{bmatrix}^T\), inoltre abbiamo \(A_1 \textbackslash b_1 = \begin{bmatrix} 0.128491620111731 \\ -2.229050279329610 \\ 2.603351955307263 \end{bmatrix}^T\), per cui possiamo dire che \(A_1 \textbackslash b_1 \approx x_1\)
\\
\\
\\
\\

\noindent Per il secondo esempio invece sono stati scelti i seguenti parametri di ingresso:
\[
A_2 = \begin{pmatrix} 1 & 2 \\ 3 & 4 \\ 5 & 6 \end{pmatrix} \quad b_2 = \begin{bmatrix} 4\\4\\6 \end{bmatrix}
\]

\noindent La funzione implementata restituisce questa soluzione: \(x_2=\begin{bmatrix} 2.666666666666669 \\ 3.166666666666670 \end{bmatrix}^T\), inoltre abbiamo \(A_2 \textbackslash b_2 = \begin{bmatrix} 2.666666666666673 \\ 3.166666666666672 \end{bmatrix}^T\), quindi anche in questo caso possiamo dire che \(A_1 \textbackslash b_2 \approx x_2\).
\\
\\
\\
\noindent Il codice Matlab usato per realizzare quanto sopra \'e il seguente.
\\

\lstinputlisting[language=Matlab]{cap3/3_9.m}
