\begin{center}
\footnotesize\noindent\fbox{
	\parbox{\textwidth}{
	Scrivere una function Matlab che, avendo in ingresso un vettore b contenente i termini noti del sistema lineare \(Ax = b\) con \textit{A} sdp e l'output dell'Algoritmo 3.6 del libro (matrice \(A\) riscritta nella porzione triangolare inferiore con i fattori \(L\) e \(D\) della fattorizzazione \(LDL^T\) di \(A\)), ne calcoli efficientemente la soluzione.
	}
}\end{center}

\noindent Il seguente codice Matlab soddisfa la richiesta. Riguardo alla function \lstinline[language=Matlab]{sist_triang_inf}, la sua implementazione \'e quella dell'Esercizio 1.
\\
\lstinputlisting[language=Matlab]{cap3/3_3.m}

\noindent Si notino i seguenti aspetti del codice mostrato:
\begin{itemize}

\item il vettore delle soluzioni dei sottosistemi viene memorizzato nelle stesse locazioni di memoria del vettore dei termini noti avuto in input;
\item la scomposizione della matrice A nei suoi fattori L e D non viene memorizzata fisicamente in due matrici diverse, ma avviene nella stessa porzione di memoria occupata dalla sola matrice A.
\item gli algoritmi scelti per la risoluzione dei sistemi triangolari accedono entrambi agli elementi della matrice \textit{per colonne}, in accordo con il tipo di memorizzazione delle matrici prevista da Matlab.
\end{itemize}

\noindent Quindi, la soluzione proposta \'e efficiente sia per quanto riguarda l'occupazione di memoria che per la minimizzazione del tempo di input/output.
