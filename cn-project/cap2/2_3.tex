\begin{center}
\footnotesize\noindent\fbox{
	\parbox{\textwidth}{
Costruire una seconda tabella analoga alla precedente relativa ai metodi di Newton, di Newton modificato e di accelerazione di Aitken applicati alla funzione polinomiale \(P\) a partire dal punto di innesco \(x_0=0\). Commentare i risultati riportati in tabela.
	}
}\end{center}

\noindent Si faccia ancora una volta riferimento ad \(f(x)\) relativa a \(P(x)=x^3 - 4x^2 + 5x - 2\) con radici in \(x_1=1\) e \(x_2=2\). Si presenta di seguito una tabella analoga, alla precedente, riferita ai metodi di Newton, di Newton modificato e di Aitken, lanciati con punto di innesco \(x_0 = 0\). Riguardo al metodo di Newton modificato, \'e stato lanciato con coefficiente del termine di correzione pari alla molteplicit\'a della radice verso la quale si converge, ovvero \(m = 2\).\\

\small\begin{tabular}{l*{20}{c}}
 & \vline& \textbf{Newton} & & & \vline& \textbf{N. Mod.} & & & \vline& \textbf{Aitken} \\
 \(tolx\) & \vline& \(x_1\)   & it. & val. & \vline& \(x_1\)   & it. & val.  &\vline& \(x_1\)  & it.& val. \\
\hline
 1/10                   & \vline&  0.8960 & 4	& 4+4  & \vline& 0.9961 & 3 & 3+3 & \vline& 1.0006 & 3  & 5+5\\
 1/50                   & \vline&  0.9859 & 4	& 7+7  & \vline& 1.0000 & 4 & 4+4 & \vline& 1.0000 & 4  & 7+7\\
 1/100                   & \vline& 0.9929 & 8	& 8+8  & \vline& 1.0000 & 4 & 4+4 & \vline& 1.0000 & 4  & 7+7\\
 1/500                   & \vline& 0.9982 & 10	& 10+10& \vline& 1.0000 & 5 & 5+5 & \vline& 1.0000 & 4  & 7+7\\
 \(1 \times 10^{-3}\) & \vline&    0.9991 & 11	& 11+11& \vline& 1.0000 & 5 & 5+5 & \vline& 1.0000 & 4  & 7+7\\
 \(2 \times 10^{-3}\) & \vline&    0.9996 & 12	& 12+12& \vline& 1.0000 & 5 & 5+5 & \vline& 1.0000 & 5  & 9+9\\
 \(5 \times 10^{-3}\) & \vline&    0.9999 & 14	& 14+14& \vline& 1.0000 & 5 & 5+5 & \vline& 1.0000 & 5  & 9+9\\
 \(1 \times 10^{-4}\) & \vline&    0.9999 & 15	& 15+15& \vline& 1.0000 & 5 & 5+5 & \vline& 1.0000 & 5  & 9+9\\
 \(1 \times 10^{-5}\) & \vline&    1.0000 & 17	& 17+17& \vline& 1.0000 & 5 & 5+5 & \vline& 1.0000 & 5  & 9+9\\
 \(1 \times 10^{-6}\) & \vline&    1.0000 & 21 	& 21+21& \vline& 1.0000 & 6 & 5+5 & \vline& 1.0000 & 5  & 9+9\\
\end{tabular} \\
\\

\normalsize\noindent Come era prevedibile, il metodo di Newton standard converge linearmente invece che quadraticamente verso la soluzione a causa della molteplicit\'a della radice in \(x_1=1\); il metodo di Newton modificato ha invece ripristinato la convergenza quadratica.
\\
\\
\noindent Anche il metodo di accelerazione di Aitken ha performato in modo simile al metodo di Newton modificato, ma la natura \textit{a due livelli} della sua funzione di iterazione ha imposto un costo maggiore riguardo al numero di valutazioni di \(f(x)\) ed \(f'(x)\) necessarie.
\\
\\
\noindent Di seguito le implementazioni Matlab dei metodi numerici utilizzati. Per il metodo di Newton semplice \'e stata ovviamente sfruttata l'implementazione usata nell'esercizio precedente.
\\
\lstinputlisting[language=Matlab]{cap2/es3_functions.m}
