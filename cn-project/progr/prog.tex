\documentclass[a4paper, 12pt]{article}

\usepackage[italian]{babel}
\usepackage[T1]{fontenc}
\usepackage[utf8]{inputenc}
\usepackage{color}

\begin{document}

\noindent Legenda:
	\begin{itemize}
		\item menzionato esplicitamente
		\color{blue} \item{dedotto dal programma dell'altro docente}
		\color{green} \item {dedotto dagli esercizi}
	\end{itemize}

\section{Capitolo 1 - prof.ssa Sestini (Moodle)}

Errori ed aritmetica finita.

\subsection{Tipi di errori}
	\begin{itemize}
		\item Errori di discretizzazione
		\item Errori di convergenza
		\item \color{blue}{Errori di round-off}
	\end{itemize}

\subsection{Rappresentazione sul calcolatore}
	\begin{itemize}
		\item Numeri interi
		\item Numeri reali di macchina
		\item Funzione floating
		\item Precisione di macchina
		\item Standard IEEE 754
	\end{itemize}

\subsection{Condizionamento}
	\begin{itemize}
		\item Definizione di "condizionamento"
		\item Condizionamento delle 4 operazioni
	\end{itemize}

\newpage

\section{Capitolo 2 - prof.ssa Sestini (Moodle)}

Ricerca delle radici di una funzione.

\subsection{Metodi iterativi --- Algoritmi}
	\begin{itemize}
		\item Bisezione
		\item Newton
		\item Accelerazione di Aitken
		\item Newton modificato e quasi Newton
	\end{itemize}

\subsection{Metodi iterativi --- Analisi}
	\begin{itemize}
		\item \color{blue}{Criteri di arresto} \color{black}
		\item Condizionamento
		\item Ordine di convergenza
		\item \color{blue}{Convergenza locale} \color{black}
		\item \color{blue}{Comportamento in caso di radici multiple} \color{black}
		\item Teorema del punto fisso
	\end{itemize}

\vspace{0.5cm}

\section{Capitolo 3 - prof.ssa Sestini (Moodle)}

Risoluzione sistemi lineari.

\subsection{Risoluzione dei casi semplici}
	\begin{itemize}
		\item Sistemi lineari diagonali
		\item Sistemi lineari ortogonali
		\item Sistemi lineari triangolari
	\end{itemize}

\subsection{Risoluzione tramite la fattorizzazione della matrice}
	\begin{itemize}
		\item Fattorizzazione $LU$
		\subitem \color{blue}{costo computazionale} \color{black}
		\item \color{blue}{Matrici a diagonale dominante} \color{black}
		\item Fattorizzazione $LDL^T$ per matrici $sdp$
		\item Pivoting
		\item \color{blue}{Studio del condizionamento} \color{black}
		\item Matrici elementari di Householder
		\subitem \color{blue}{fattorizzazione $QR$} \color{black}
	\end{itemize}

\subsection{Altri metodi}
	\begin{itemize}
		\item Sistemi lineari sovradeterminati
		\item \color{blue}{Metodo di Jacobi} \color{black}
		\item \color{blue}{Metodo di Gauss-Seidel} \color{black}
		\item \color{blue}{Splitting regolari di matrici} \color{black}
	\end{itemize}

\vspace{0.5cm}

\section{Capitolo 4 - prof. Brugnano}

Approssimaione di funzioni.

\subsection{Interpolazione polinomiale}
	\begin{itemize}
		\item Forma di Lagrande
		\item Forma di Newton
		\item Errore nell'interpolazione
		\item Condizionamento del problema
		\item Ascisse di Chebishev
		\item \color{green}{Polinomio interpolante di Hermite} \color{black}
	\end{itemize}

\subsection{Interpolazione con Spline}
	\begin{itemize}
		\item \color{green}{Spline cubica naturale} \color{black}
		\item \color{green}{Spline cubica not-a-knot} \color{black}
	\end{itemize}

\subsection{Approssimazione polinomiale ai \textit{minimi quadrati}}
	\begin{itemize}
		\item Metodo ed esempio
	\end{itemize}

\vspace{0.5cm}

\section{Capitolo 5 - prof. Brugnano}

Formule di quadratura.

\subsection{Formule di Newton-Cotes}

	\begin{itemize}
		\item Errore
		\item Formule composite
		\subitem \color{green}{Formula dei Trapezi} \color{black}
		\subitem \color{green}{Formula di Simpson} \color{black}
		\item Formule adattive
	\end{itemize}

\vspace{0.5cm}

\section{Capitolo 6 - prof. Brugnano}

Metodi per la ricerca degli autovalori di una matrice.

	\begin{itemize}
		\item Metodo delle potenze
		\subitem Applicazione al Google pagerank
	\end{itemize}

\end{document}
