Dato \(P(x) = x^3 - 4x^2 + 5x - 2\), si determinano anzitutto le sue radici analiticamente:

\[
P(x) = x^3 - 4x^2 + 5x - 2\ = (x-2)(x-1)^2
\]

\noindent Gli zeri di \(P(x)\) sono quindi \(x_1=1\) con molteplicit\'a \(2\) e \(x_2=2\) con molteplicit\'a \(1\).\\

\noindent Il metodo di bisezione \'e utilizzabile in quanto le ipotesi necessarie per il suo impiego sono soddisfatte:

\begin{itemize}

\item \(f(x)\) associata a \(P(x)\) \'e definita e continua nell'intervallo \([0,3]\)
\item \(f(0)f(3) = (-2)(4) = -8 < 0\)

\end{itemize}

\noindent Per conoscere a quale zero di \(P(x)\) tender\'a la successione generata dal metodo di bisezione, basta eseguire manualmente la prima iterazione:

\begin{itemize}
\item \(f(\frac{0+3}{2}) = -\frac{1}{8}\)
\item \(f(0)f(\frac{3}{2}) = \frac{1}{4}\)
\item \(f(\frac{3}{2})f(3) = -\frac{1}{2}\)
\end{itemize}

\noindent Si prosegue quindi la bisezione nell'intervallo \([\frac{3}{2}, 3]\), perci\'o si tender\'a allo zero in \(x_2=2\).

\noindent Si applica adesso il metodo di bisezione con il seguente nella sua interezza e si mostra il numero di iterazioni necessarie al variare della soglia di tolleranza \(tolx\).\\

\begin{tabular}{l*{6}{c}}
 \(tolx\) & appross. \(x_2\) & iterazioni \\
\hline
 1/10 & 1.8750 & 3\\
 1/20 & 1.9688 & 5\\
 1/50 & 2.0156 & 6\\
 1/100 & 1.9922 & 7\\
 1/500 & 1.9980 & 9\\
 \(1 \times 10^{-3}\) & 2.0010 & 10\\
 \(2 \times 10^{-3}\) & 1.9995 & 11\\
 \(5 \times 10^{-3}\) & 1.9999 & 13\\
 \(1 \times 10^{-4}\) & 2.0001 & 14\\
 \(1 \times 10^{-5}\) & 2.0000 & 17\\
 \(1 \times 10^{-6}\) & 2.0000 & 20\\
\end{tabular} \\

\noindent La precedente tabella \'e stata riempita in riferimento alla seguente implementazione del metodi di bisezione, descritta in codice Matlab.\\

\lstinputlisting[language=Matlab]{cap2/es1.m}

\noindent Si noti che non \'e stato inserito alcuna condizione di uscita dal ciclo \(while\) dato che gli zeri del polinomio da studiare sono stati determinati analiticamente ed il comportamento del metodo \'e stato previsto. Il codice \'e stato semplificato e non si \'e corso alcun rischio di non veder terminare l'algoritmo.
