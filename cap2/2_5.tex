\begin{center}
\footnotesize\noindent\fbox{
	\parbox{\textwidth}{
Definire una procedura iterativa basata sul metodo delle secanti sempre per approssimare \(\sqrt\alpha \), per un assegnato \(\alpha > 0\). completare la tabella precedente aggiungendovi i risultati ottenuti con tale precedura partendo da \(x_0=5\) e \(x_1=3\). Commentare i risultati riportati in tabella.
	}
}\end{center}

\begin{tabular}{l*{6}{c}}
 & \vline& \textbf{Newton} & & \vline& \textbf{Secanti}\\
 it. & \vline& appr. \(\sqrt\alpha \) & \(\Delta\sqrt\alpha \) & \vline& appr. \(\sqrt\alpha \) & \(\Delta\sqrt\alpha \) \\
\hline
 1 & \vline& 3      & 0.7639				& \vline& 2.5000 & 0.2639 \\
 2 & \vline& 2.3333 & 0.0973				& \vline& 2.2727 & 0.0367 \\
 3 & \vline& 2.2381 & 0.0020				& \vline& 2.2381 & 0.0020 \\
 4 & \vline& 2.2361 & 9.1814 \(\times10^{-7}\) 	& \vline& 2.2361 & 1.6475 \(\times10^{-5}\) \\
 5 & \vline& ''     & 1.8829 \(\times10^{-13}\)	& \vline& ''     & 7.4651 \(\times10^{-9}\) \\

\end{tabular} \\

