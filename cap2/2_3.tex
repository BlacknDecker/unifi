Si faccia ancora una volta riferimento ad \(f(x)\) relativa a \(P(x)=x^3 - 4x^2 + 5x - 2\) con radici in \(x_1=1\) e \(x_2=2\). Si presenta di seguito una tabella analoga, alla precedente, riferita ai metodi di Newton, di Newton modificato e di Aitken, lanciati con punto di innesco \(x_0 = 0\).\\

\begin{tabular}{l*{12}{c}}
 & \vline& \textbf{Newton} & & \vline& \textbf{N. Mod.} & & \vline& \textbf{Aitken} \\
 \(tolx\) & \vline& appr. \(x_1\) & it. & \vline& appr. \(x_1\) & it.& \vline& appr. \(x_1\) & it.\\
\hline
 1/10 & \vline& 0.8960 & 3\\
 1/20 & \vline& 0.9457 & 4\\
 1/50 & \vline& 0.9859 & 6\\
 1/100 & \vline& 0.9929& 7\\
 1/500 & \vline& 0.9982& 9\\
 \(1 \times 10^{-3}\) & \vline& 0.9991 & 10\\
 \(2 \times 10^{-3}\) & \vline& 0.9996 & 11\\
 \(5 \times 10^{-3}\) & \vline& 0.9999 & 13\\
 \(1 \times 10^{-4}\) & \vline& '' & 14\\
 \(1 \times 10^{-5}\) & \vline& 1.000 & 17\\
 \(1 \times 10^{-6}\) & \vline& '' & 20\\
\end{tabular} \\
