Si faccia ancora una volta riferimento ad \(f(x)\) relativa a \(P(x)=x^3 - 4x^2 + 5x - 2\) con radici in \(x_1=1\) e \(x_2=2\). Si presenta di seguito una tabella analoga, alla precedente, riferita ai metodi di Newton, di Newton modificato e di Aitken, lanciati con punto di innesco \(x_0 = 0\).\\

\begin{tabular}{l*{12}{c}}
 & \vline& \textbf{Newton} & & \vline& \textbf{N. Mod.} & & \vline& \textbf{Aitken} \\
 \(tolx\) & \vline& appr. \(x_1\) & it. & \vline& appr. \(x_1\) & it.& \vline& appr. \(x_1\) & it.\\
\hline
 1/10 & \vline& 0.8960 & 3						& \vline& 0.9961 & 2 & \vline& 1.0006 & 2 \\
 1/20 & \vline& 0.9457 & 4						& \vline& 0.1000 & 3 & \vline& ''     & ''\\
 1/50 & \vline& 0.9859 & 6						& \vline& ''     & 3 & \vline& 1.0000 & 3 \\
 1/100 & \vline& 0.9929& 7						& \vline& ''     & 3 & \vline& ''     & ''\\
 1/500 & \vline& 0.9982& 9						& \vline& ''     & 4 & \vline& ''     & ''\\
 \(1 \times 10^{-3}\) & \vline& 0.9991 & 10		& \vline& ''     & 4 & \vline& ''     & ''\\
 \(2 \times 10^{-3}\) & \vline& 0.9996 & 11		& \vline& ''     & 4 & \vline& ''     & 4\\
 \(5 \times 10^{-3}\) & \vline& 0.9999 & 13		& \vline& ''     & 4 & \vline& ''     & ''\\
 \(1 \times 10^{-4}\) & \vline& '' & 14			& \vline& ''     & 4 & \vline& ''     & ''\\
 \(1 \times 10^{-5}\) & \vline& 1.000 & 17		& \vline& ''     & 4 & \vline& ''     & ''\\
 \(1 \times 10^{-6}\) & \vline& '' & 20 		& \vline& ''     & 5 & \vline& ''     & ''\\
\end{tabular} \\

\noindent Come era prevedibile, il metodo di Newton standard converge linearmente invece che quadraticamente verso la soluzione a causa della molteplicit\'a della radice verso la quale si converge; il metodo di Newton modificato ha invece ripristinato la convergenza quadratica. DUE PAROLE SU AITKIN

\noindent Di seguito le implementazioni Matlab dei metodi numerici utilizzati. Per il metodo di Newton \'e stato sfruttata l'implementazione usata nell'esercizio precedente.

\lstinputlisting[language=Matlab]{cap2/es3_functions.m}
