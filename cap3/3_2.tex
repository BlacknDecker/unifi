\begin{center}
\footnotesize\noindent\fbox{
	\parbox{\textwidth}{
	Utilizzare l'Algoritmo 3.6 del libro per stabilire sele seguenti matrici sono sdp o no:

\[
A_1 = \begin{pmatrix}1 & -1 & 2 & 2 \\ -1&5&-14&2\\ 2&-14&42&2\\2&2&2&65 \end{pmatrix},
A_2 = \begin{pmatrix}1 & -1 & 2 & 2 \\ -1&6&-17&3\\ 2&-17&48&-16\\2&3&-16&4 \end{pmatrix}
\]
	}
}\end{center}

\noindent Il seguente codice Matlab implementa una versione dell'algoritmo 3.6 del libro e la utilizza per rispondere a quanto richeisto. Nel merito, la matrice \(A_1\) \'e simmetrica e definita positiva, mentre la matrice \(A_2\) non lo \'e.

\lstinputlisting[language=Matlab]{cap3/3_2.m}
