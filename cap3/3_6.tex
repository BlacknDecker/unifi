\begin{center}
\footnotesize\noindent\fbox{
	\parbox{\textwidth}{
	Sia \(A = \begin{pmatrix} \epsilon & 1 \\ 1 & 1 \end{pmatrix}\) con \(\epsilon=10^{-13}\).  Definire L triangolare inferiore a diagonale unitaria e U triangolare superiore in modo che il prodotto LU sia la fattorizzazione LU di A e, posto \(b=Ae\) con \(e=(1,1)^{T}\), confrontare l'accuratezza della soluzione che si ottiene usando il comando \(U \textbackslash (L \textbackslash b)\) (Gauss senza pivoting) e il comando \(A \textbackslash b\) (Gauss con pivoting).
	}
}\end{center}

\noindent La rappresentazione in una sola matrice della fattorizzazione LU di A, ottenuta mediante l'applicazione dell'algoritmo 3.5 del libro, \'e la seguente:
\[
A_{LU} = \begin{pmatrix} 10^{-13} & 1 \\ 10^{13} & -10^{13}\end{pmatrix}
\]

\noindent Le due matrici L ed U quindi sono:

\[
L = \begin{pmatrix} 1 & 0 \\ 10^{13} & 1\end{pmatrix}\]
\[
U = \begin{pmatrix} 10^{-13} & 1 \\ 0 & -10^{13}\end{pmatrix}
\]
\\
\noindent Riguardo l'altra richiesta, si sono verificati i seguenti fatti:

\[
b = Ae = \begin{pmatrix} \epsilon & 1 \\ 1 & 1 \end{pmatrix} \begin{bmatrix} 1 \\ 1\end{bmatrix} = \begin{bmatrix} \approx 1 \\ 2 \end{bmatrix}
\]
\[
U \textbackslash (L \textbackslash b) = \begin{bmatrix} 0.992 \\ 1\end{bmatrix}
\]
\[
A \textbackslash b = \begin{bmatrix} 1 \\ 1\end{bmatrix}
\]

\noindent Si pu\'o notare come il vettore \(b\) calcolato col metodo di Gauss senza pivoting abbia un'accuratezza minore rispetto alla sua versione calcolata col metodo di Gauss con pivoting.
\\
\\In generale, i problemi che sorgono in quel caso sono dati tutti dal fatto che \(\epsilon \) \'e un numero molto piccolo. La condizione necessaria per l'esistenza dell procedimento di \textit{eliminazione di Gauss} senza \textit{pivoting} (cio\'e, al passo \textit{i}-esimo, \(a_{i,i} \neq 0\)) \'e molto vicina al venire meno gi\'a alla prima ed unica iterazione, per la quale abbiamo \(a_{1,1} = \epsilon \approx 0\).
\\

\noindent A causa di questo, succede che la matrice \(L_1A\), con \(L_1\) la prima --- ed unica --- \textit{matrice elementare di Gauss}, \'e molto vicina all'essere singolare.

\[
L_1 = \begin{pmatrix} 1 & 0 \\ -\frac{1}{\epsilon} & 1 \end{pmatrix} \quad L_1A = \begin{pmatrix} \epsilon & 1 \\ 0 & 1\end{pmatrix} \quad \det(L_1A) \approx 0
\]
\\
Persino Matlab restituisce un'avvertimento:
\lstinputlisting[language=Matlab]{cap3/warning.m}

\noindent Il metodo di Gauss con pivoting invece \'e studiato appositamente per funzionare senza bisogno di avere come ipotesi la condizione \(a_{i,i} \neq 0\) al passo \textit{i}-esimo, pertanto approssima in maniera soddisfacente il risultato.
\\
\\
\noindent L'intero codice Matlab usato per svolgere questo Esercizio \'e quello che segue.

\lstinputlisting[language=Matlab]{cap3/3_6.m}

