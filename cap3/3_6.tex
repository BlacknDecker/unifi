\begin{center}
\footnotesize\noindent\fbox{
	\parbox{\textwidth}{
	Sia \(A = \begin{pmatrix} \epsilon & 1 \\ 1 & 1 \end{pmatrix}\) con \(\epsilon=10^{-13}\).  Definire L triangolare inferiore a diagonale unitaria e U triangolare superiore in modo che il prodotto LU sia la fattorizzazione LU di A e, posto \(b=Ae\) con \(e=(1,1)^{T}\), confrontare l'accuratezza della soluzione che si ottiene usando il comando \(U \textbackslash (L \textbackslash b)\) (Gauss senza pivoting) e il comando \(A \textbackslash b\) (Gauss con pivoting).
	}
}\end{center}

\noindent La rappresentazione in una sola matrice della fattorizzazione LU di A \'e la seguente:
\[
A_{LU} = \begin{pmatrix} 10^{-13} & 1 \\ 10^{13} & -10^{13}\end{pmatrix}
\]

\noindent Le due matrici L ed U quindi sono:

\[
L = \begin{pmatrix} 1 & 0 \\ 10^{13} & 1\end{pmatrix}\]
\[
U = \begin{pmatrix} 10^{-13} & 1 \\ 0 & -10^{13}\end{pmatrix}
\]

\noindent Ed ovviamente abbiamo che:
\[
LU = A
\]
\\

\noindent Riguardo l'altra richiesta, si sono verificati i seguenti fatti:

\[
e = \begin{bmatrix} 1 \\ 1\end{bmatrix}
\]
\[
b = Ae = \begin{pmatrix} \epsilon & 1 \\ 1 & 1 \end{pmatrix} \begin{bmatrix} 1 \\ 1\end{bmatrix} = \begin{bmatrix} \approx 1 \\ 2 \end{bmatrix}
\]
\[
U \textbackslash (L \textbackslash b) = \begin{bmatrix} 0.992 \\ 1\end{bmatrix}
\]
\[
A \textbackslash b = \begin{bmatrix} 1 \\ 1\end{bmatrix}
\]

\noindent Si pu\'o notare come il vettore \(b\) calcolato col metodo di Gauss senza pivoting abbia un'accuratezza minore rispetto alla sua versione calcolata col metodo di Gauss con pivoting. Questo perch\'e essendo \(\epsilon\) un numero molto piccolo, la matrice \'e molto vicina all'essere singolare, quindi... APPROFONDIRE
\\

\noindent Proprio riguardo a questo aspetto, anche Matlab restituisce un'avvertimento:
\lstinputlisting[language=Matlab]{cap3/warning.m}

\noindent L'intero codice Matlab usato per svolgere questo Esercizio \'e quello che segue.

\lstinputlisting[language=Matlab]{cap3/3_5.m}

