\begin{center}
\footnotesize\noindent\fbox{
	\parbox{\textwidth}{
	Scrivere una function Matlab per la risoluzione di un sistema lineare con matrice dei coefficienti triangolare inferiore a diagonale unitaria. Inserire un esempio di utilizzo.
	}
}\end{center}

\noindent Si veda il seguente codice Matlab:

\lstinputlisting[language=Matlab]{cap3/3_1.m}

\noindent La funzione realizzata risolve sistemi del tipo richiesto prendendo in input la matrice dei coefficienti \(A\) ed il vettore dei termini noti \(b\), restituendo il vettore delle soluzioni.
\\
\\
\noindent Nell'esempio risolto nel codice, il sistema in oggetto era questo:
\[
\begin{bmatrix}1 & 0 & 0 \\ 2 & 1 & 0\\ -1 & 2 & 1 \end{bmatrix} \vec{x} = \begin{bmatrix}1 \\ 2 \\ -2 \end{bmatrix}
\]
\noindent Il vettore delle soluzioni calcolato dalla funzione \'e, correttamente, \(\begin{bmatrix}1 \\ 0 \\ -1 \end{bmatrix}\).
