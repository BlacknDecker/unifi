\begin{center}
\footnotesize\noindent\fbox{
	\parbox{\textwidth}{
Verificato che la funzione \(f(x_1,x_2) = x_1^2+x_2^3-x_1x_2\) ha un punto di minimo relativo in (1/12, 1/6), costruire una tabella in cui si riportano il numero di iterazioni eseguite, e la norma euclidea dell'ultimo incremento e quella dell'errore con cui viene approssimato il risultato esatto utilizzando la function sviluppata al punto precedente per valori delle tolleranze pari a \(10^{-t}\), con t = 3,6. Utilizzare (1/2, 1/2) come punto di innesco. Verificare che la norma dell'errore \'e molto pi\'u piccola di quella dell'incremento (come mai?)
	}
}\end{center}

\noindent Si verifica innanzitutto l'esistenza di un punto di minimo relativo in (1/12, 1/6). Si rinominano le variabili in x ed y per convenienza nella notazione.
\[
\frac{\partial}{\partial x}f(x,y) = 2x -y \quad \frac{\partial}{\partial y}f(x,y) = 3y^2 - x
\]
\[
\begin{cases}
2x -y = 0 \\
3y^2 - x = 0
\end{cases}
\quad \text{ha come soluzioni} \quad x=0, y=0 \quad \text{e} \quad x=\frac{1}{12}, y=\frac{1}{6}
\]
\[
H =
\begin{bmatrix} f_{xx} & f_{xy} \\ f_{yx} & f_{yy} \end{bmatrix}
=
\begin{bmatrix} 2 & -1 \\ -1 & 6y \end{bmatrix}
\quad
\det(H) = 12y -2
\]

\noindent Il determinante della matrice Hessiana \'e nullo per \(y=\frac{1}{6}\).
\[
f(\frac{1}{12}, \frac{1}{6}) = -\frac{1}{432} \quad \overline{f}(x,y) = f(x,y) + \frac{1}{432} \quad \overline{f}(\frac{1}{12}, \frac{1}{6}) = 0
\]

Matlab inequality plot per \(\overline{f}(x,y) \geq 0 \quad \)

%L'esistenza del punto di minimo relativo si pu\'o verificare facilmente plottando la funzione su Matlab lanciando il seguente codice:

% metticelo
%[x1,x2] = meshgrid(-2:.2:2);
%y =  x1^2 + x2^3 - x1*x2;
%surf(x1,x2,y);

% derivate parziali: d/dx 2x-y
% d/dy 3y^2 -x
%
% si cerca i punti stazionari col sistema delle derivate parziali uguali a zero
%
% matrice hessiana delle derivate parziali seconde
%
% si valuta l'hessiana nei punti stazionari e si fa deerminante
% determinante positivo, primo elemento positivo -> punto di min relativo
